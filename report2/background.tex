



\chapter{Background}

This chapter discusses some of the background information necessary to make sence of the
following chapters; short introductions to the fundamentals of \emph{functional programming} 
(\emph{combinatory logic} and \emph{the lambda calculus}), the \emph{Haskell} 
programming language, and \emph{virtual machines} (\emph{JIT compilation}, \emph{VM construction} 
and the \emph{PyPy project})






\section{Functional programming}

Functional programming is a programming paradigm where computation is modelled as the evaluation
of expressions. Mutability and state is avoided. Functional programming is based on the 
lambda calculus, a formal system invented by Alonzo Church.

\subsection{Combinatory logic}

...

\subsection{Lambda calculus}

In mathematical form, one
could write a function as $f : x \mapsto x^2$. However, this notation is not adequate when
higher order functions are involved, i.e. functions that return other functions. The lambda calculi
(invented by A. Church) is another formal system for function notation that allows for higher
order functions. In the lambda calculus, such a function would be written $\lambda x.x^2$, and the
lambda abstraction would be applied to a value by parenthesis. In the same way $f : x \mapsto x^2$
applied to $2$ would result in $f : 2 \mapsto 2^2$, in lambda calculus would be 
written $((\lambda x.x^2) \;\; 2)$ and reduced to $2$.

\subsection{Lambda calculus with types}


...





\section{Haskell}

% TODO: Fix the following paragraph

\emph{Haskell} is a \emph{non-strict} \emph{purely-functional} programming language. The
name is was chosen. \cite{hudak2007history}


\subsection{Syntax}

This section covers some of the main design choises made by the Haskell commity when
creating the syntax for Haskell. For a more formal definition of the syntax, 
see \cite{jones2003haskell} 

\subsubsection {Layout}

In many imperative programming languages, the semicolon is used to separate sequential
statements. However, in Haskell, the notion of sequence is completely absent due to the
fact that it is a language without side effects. \cite{hudak2007history}

Haskell exploits the physical layout of the text (in a way similar to Python) in order
to avoid unnecessary syntactic constructs (such as the use of curly braces in \emph{C}).
Haskell does however support the use of explicit curly braces and semicolons, the reason
for this being that people may want to write Haskell programs that generate Haskell programs,
and this would be easier without the need to keep track of indentation. \cite{hudak2007history}

\subsubsection{Functions and function application}

Defining functions in Haskell can be done in many different ways. It is a functional programming
language...

\begin{itemize}

\item{Currying}, a function taking two arguments can be defined by a function taking one argument,
and returning a function taking the second argument. Haskell supports both curried and uncurried
function definitions. Function application is left associative, so $f\;x\;y$ is applied $(f\;x)\;y$.
\cite{hudak2007history}

\item{Anonymous functions} are syntactically similar to the lambda calculus:
represented by the \inhas{\textbackslash} symbol since it is the ASCII character
most similar to the $\lambda$:  \inhas{\textbackslash x->exp}. 
The \inhas{->} is used however, since the period is reserved
for function composition. \cite{hudak2007history}

\item{Prefix operators}; Haskell has only a single prefix operator defined. This is the arithmetic
negation operator.\cite{hudak2007history}

\item{Infix operators} are very important in Haskell, as it is meant to look as much like mathematics
as possible. Infix operators are also definable by the user, including precedence and associativity.
The relationship between infix application and conventional function application is simple; infix
application always have precedence. This means that the expression 
\inhas{f x + g y} never has
to be parenthesized.\cite{hudak2007history}

\item{Sections} are partial applications of infix operators to no arguments, both the left argument and
the right. By surrounding the result in parenthesis. 
Since all values in Haskell are "first class", especially functions, 
this solves the problem of what is meant by the
expression \inhas{f + x}, without sections, this could mean that 
\inhas{+} takes the arguments \inhas{f} and 
\inhas{x},
or that \inhas{f} takes the arguments \inhas{+} 
and \inhas{x}. This also gives the ability to partially apply infix
operators. Ordinary functions can also be applied infix by surrounding the function name in single
quotes; \inhas{x 'f' y} is the same as \inhas{f x y}. 
\cite{hudak2007history}

\end{itemize}

\subsubsection{Namespaces and keywords}

Namespaces in Haskell are designed to give the user as much freedom as possible, while avoiding
ambiguity. Therefore, a set of lexemes are defined for each namespace in such a way that this,
together with the context, is enough to determine which namespace a keyword refers to. For example,
in the following, \inhas{Vector} can refer to both a type constructor and a data constructor:

\lstset{language=Haskell}
\begin{lstlisting}
data Vector = Vector Float Float
\end{lstlisting}

However, it is clear from the context that the leftmost reference is in the namespace of type
constructors, and the rightmost in the namespace of data constructors.

Haskell also allows for shadowing of variebles, i.e., reusing a variable name in a deeper 
scope.

\subsubsection{Declaration style vs. expression style}

There are two main styles in which a functional program can be written in Haskell: 
\emph{declaration style} or \emph{expression style}. 

The declaration style uses pattern matching and guards, and multiple equations
to define a function. In the expression style, a function is defined by composing it from smaller 
expressions. Each style can be characterised by a small set of syntactic constructs. 

The declaration style constructs beeing: \inhas{where} clause, function arguments on left hand side, 
pattern matching in function definitions and guards on function definitions. 
And the expression style constructs beeing:
\inhas{let} expressions, lambda expressions, \inhas{case} expressions and \inhas{if} expressions

As opposed to Miranda, which placed guards on the right side of expressions 
(similar to mathematic notation), in Haskell guards are put on the left side. Also,
in Haskell, a where clause is attached to a declaration, not an expression. This means that
it scopes over guards as well as the right-hand side of declarations.\cite{hudak2007history}


\subsubsection{List comprehensions}

List comprehensions are very usefull;

\begin{lstlisting}
[ x*x | x <- xs ]
\end{lstlisting}

\subsubsection{Comments}

...

\subsection{Algebraic data types}

\subsection{Type system and type classes}

\subsection{I/O and monads}

\subsection{Modules, packages and the Foreign Function Interface}







\section{Virtual machines}

\subsection{Just-in-time compilation}

... ...

\subsection{The PyPy project}

The PyPy (Python in Python) project aims to implement a Python interpreter in Python.
To do this, a restricted proper subset of Python called RPython has been designed.
The reason for using RPython is that type inference can be performed on it. Type
inference is the act of figuring out what type a variable has. In Python, a variable
can have many different types during it's lifespan, and it is therefore not possible
to perform type inference on it. However, in RPython, a variable can only have a 
single type during it's lifespan. This makes it easier to translate RPython to a lower
level language, like C. This is in fact what PyPy does, a Python interpreter is written
in RPython, and then translated into C. Although PyPy can translate the RPython programs
directly into C, it can also include a JIT compiler in this translation, and it has
a number of garbage collectors ready in the GC framework.



Functional programming languages is a paradigm currently
unexplored by the PyPy project. The introduction of Haskell
to the PyPy project will eventualy inspect wether or not 
the methods used in the PyPy project are applicable to languages
like Haskell.


