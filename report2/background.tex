
\chapter{Background}


Introduce imperative programming languages

...

Introduce declarative/functional programming languages as 



\section*{Haskell}

Haskell is a purely functional programming language. A functional programming language
can be characterized as evaluating expressions. As opposed to an imperative 
programming language, where a program is described as a sequence of statements.



\section*{PyPy}

The PyPy (Python in Python) project aims to implement a Python interpreter in Python.
To do this, a restricted proper subset of Python called RPython has been designed.
The reason for using RPython is that type inference can be performed on it. Type
inference is the act of figuring out what type a variable has. In Python, a variable
can have many different types during it's lifespan, and it is therefore not possible
to perform type inference on it. However, in RPython, a variable can only have a 
single type during it's lifespan. This makes it easier to translate RPython to a lower
level language, like C. This is in fact what PyPy does, a Python interpreter is written
in RPython, and then translated into C. Although PyPy can translate the RPython programs
directly into C, it can also include a JIT compiler in this translation, and it has
a number of garbage collectors ready in the GC framework.



Functional programming languages is a paradigm currently
unexplored by the PyPy project. The introduction of Haskell
to the PyPy project will eventualy inspect wether or not 
the methods used in the PyPy project are applicable to languages
like Haskell.


\section*{Similar work}
