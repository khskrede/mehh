



\chapter{Haskell-Python Core' interpreter}

Haskell-Python is an interpreter for a language similar to Core, we call it Core' here.
For a class-diagram, see .....

\section*{Classes}

All base classes inherit from $object$.

\subsection{Symbol}

A cached symbol that can be compared by identity (which is not true for strings).

\subsection{HaskellObject}

Base class for objects that the interpreter handles.

\subsubsection{Value}

\begin{itemize}
\item{$Constructor$} A constructor. This is an abstract base class, subclasses are 
generated in ConstructorN for various arguments.
\item{$AbstractFunction$} The $AbstractFunction$ is a base class for functions. The 
$Function$ and $PrimFunction$ classes inherits it. The $Function$ class describes
user-defined functions (i.e. written in haskell). The $PrimFunction$ class describes
primitive functions (i.e. a function not implemented in Core', but at the machine level)
\item{Other Values}
\end{itemize}

\subsubsection{Substitution}

\subsubsection{Var}

\subsubsection{NumberedVar}

\subsubsection{Application}

\subsubsection{Thunk}

\subsection{Rule}

One rule of a user-defined function.

\subsection{StackElement}

\subsubsection{CopyStackElement}

\subsubsection{UpdateStackElement}

\section{Global functions}

\subsection{$make\_arg\_subclass$}

\subsection{$make\_constructor$}

Creates a new constructor.

\subsection{$constr$}

\subsection{$enum$}

\subsection{$function$}


\subsection{$make\_application$}


\subsection{The $evaluate\_hnf$} function simply asserts that the object to be evaluated
is an $Application$, and then calls $main\_loop$

\subsection{$main\_loop$}

\section{Primitives}




\section{Extensions}

\subsection{JSCore parser}

\subsection{Partial function application}

The following extensions was made to the Core' interpreter to aid in the conversion from
Core to Core':

\begin{itemize}
\item Added a function $make\_partial\_app$ that takes a $PartialApp$ and a argument, and returns
a $PartialApp$ if the numbers of arguments collected does not match the number of arguments required,
otherwise, it returns an $Application$
\end{itemize}


\subsection{GHC Haskell libraries}



\subsection{Primitives}


