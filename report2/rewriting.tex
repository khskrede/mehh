
\chapter{Core to Core' mapping}
\label{chap:rewrite}

An in-depth description of the intermediate languages and the mapping from
Core to Core' is presented in this chapter. 
Starting with a description of the Core language, and continuing with a 
description of the parsing. 

\section{Core}

\subsection*{External representation}

The Core language is an intermediate language used by GHC. External-Core
is an external representation of this language. See appendix 
\ref{coregrammar} for a formal grammar of External-Core. For this project,
another format was defined, called JSCore. JSCore is a version of Core
very similar to External-Core, except that it is also in JSON format. 
See appendix \ref{jsongrammar} for a formal definition of JSON, and 
appendix \ref{jscoregrammar} for a formal definition of JSCore. This
chapter discusses how JSCore is parsed and turned into Core'. The Core
language is referred to as System $F_C$ when discussing it's theory.
Following is a short discussion of System $F_C$ and some of it's 
properties that are relevant to this discussion.

\subsection*{System $F_C$}

System $F$ is a typed lambda calculus, very popular as an intermediate language
for functional language compilers. System $F_C$ is a super-set of System $F$ 
that uses explicit type-equality coercions as witnesses to justify explicit
type-cast operations. System $F_C$ is currently GHCs intermediate language.
\cite{sulzmann2007system}

Both types and coercions are eliminated before running the program, so they have
no operational effect and no run-time cost.
\cite{sulzmann2007system}

No compilers use a pure System $F$ as its intermediate language as this would
require very heavy encoding. Most compilers extend System $F$ with algebraic
data types, data constructors and case-expressions.
\cite{sulzmann2007system}

Although Cast coercions have no operational effect, they serve to help the
GHC type-system.

%\subsubsection*{Classes}
%\label{subsec:classes}

A Haskell type-class is generally turned into System $F$ by creating a 
record type for each class, called a dictionary, containing the class methods.
Each instance is converted into a dictionary value, and passing such 
dictionaries to the generic functions mentioning a class in its signature,
lets the generic function find which specific function to apply.
\cite{sulzmann2007system}

%\subsubsection*{Type erasure}

In System $F_C$, evidence is passed around for type equalities. However,
since this evidence are represented in the form of types, and types are
erased before running the program, this evidence has no effect.
\cite{sulzmann2007system}

\section{Parsing}

The following content is an introduction to the actual implementation
of the parser without becoming too technical. Its purpose is to communicate
the similarities and differences between JSCore 
and Core', as well as defining a mapping from the former to the latter.

The most notable constructs of JSCore are described and it is shown how they
are converted into Core'. The focus is on some examples of JSCore pseudo code.
Pseudo code is used for practical reasons, as real code is too large to be 
presented.

The following content utilizes a function to describe the mapping:

\begin{equation*}
ToCore'(stack, construct)
\end{equation*}

which is defined to take a stack of variables and a piece of JSCore code 
as arguments and return Core' code. 

Variables are added to the stack with the concatenation operator (":"). 
So, the code 
\lstset{ %
language=Haskell,
label=lst:int1
}
\begin{lstlisting}
stack:variable 
\end{lstlisting}
simply returns the original $stack$ with $variable$ appended to it.

\subsection*{Packages, Modules and Definitions}

GHC deals with packages by simply naming the loaded module by its full identifier
name, which includes the package name. This means that it is not necessary to deal
with the package system. It suffices to resolve what package to load from the name
of the module.

When we parse a module, we create a object of type Module. This object is
then put in a dictionary, where we can look it up by its full identifier name.
The module object contains dictionaries of all the definitions defined in it.

So, the function call:
\lstset{ %
language=Haskell,
label=lst:int1
}
\begin{lstlisting}
ToCore'(stack, {"aexp": ["ghc-prim:GHC.Types"], [Int]} )" 
\end{lstlisting}

will return the function placed under the name "Int" in the module "ghc-prim:GHC.Types" 
where "ghc-prim" refers to the package containing the module. 

The function 
\lstset{ %
language=Haskell,
label=lst:int1
}
\begin{lstlisting}
ToCore'(stack, {"qvar": "func1", "ty": ..., "exp": exp } )
\end{lstlisting}
will
put the result of parsing "exp" in the module being parsed under the name "func1".

\subsection*{Lambda abstraction}

In the following example, the names "sometype" and "anexpression" refers to a 
type constructor and an expression in JSCore format respectively.

\begin{figure}[H]
\lstset{ %
language=Haskell,
caption=Python class implementing the Haskell Int Value.,
label=lst:int1
}
\begin{lstlisting}
{"lambda" : {"vbind" : "var1", "ty" : {"qtycon" : "sometype"} }
 "exp" : anexpression }
\end{lstlisting}
\end{figure}

A lambda abstraction is defined as a JSON object containing two pairs, namely 
"lambda" and "exp". The "lambda" pair defines a variable that is in scope in the
expression contained in the "exp" pair.

Since the Core' expressions contained in Core' functions only have variables in 
scope that are declared in the function, we keep a stack of variables as we traverse
the JSCore program. When we encounter this specific lambda abstraction, the variable
"var1" is added to the stack, and the expression is turned into the following Core' code. 
Note that the variable-stack is used by our $ToCore'()$ function, and when
the function returns, the new variable is popped from the stack.

\begin{figure}[H]
\lstset{ %
language=Haskell,
caption=Lambda abstraction in Core' format,
label=lst:int1
}
\begin{lstlisting}
func( "name", [
    rule(stack:var1 , ToCore'(stack:var1, anexpression) )
])
\end{lstlisting}
\end{figure}

Also note that the "sometype" placeholder is not used, as variables in Core' does
not have a type. If the "anexpression" placeholder is also a lambda-abstraction, the
expression simply becomes:

\begin{figure}[H]
\lstset{ %
language=Haskell,
caption=Lambda abstraction in Core' format,
label=lst:int1
}
\begin{lstlisting}
ToCore'(stack:var1, anexpression)
\end{lstlisting}
\end{figure}

Which becomes:

\begin{figure}[H]
\lstset{ %
language=Haskell,
caption=Lambda abstraction in Core' format,
label=lst:int1
}
\begin{lstlisting}
func( "name", [
    rule(stack:var1:var2 , ToCore'(stack:var1:var2, subexpression) )
])
\end{lstlisting}
\end{figure}

where "subexpression" is a placeholder for the JSCore expression contained in the 
lambda-abstraction of "anexpression".


\subsection*{Case expressions and alternatives}

The most complicated construct in Core is the case-expression. A case-expression
in Core can contain three different types of alternatives. A literal alternative,
a constructor alternative or a default alternative. 

Following is an excerpt from appendix \ref{jscoregrammar} defining the syntax 
of case expressions (see appendix \ref{jscoregrammar} for an explanation of the
notation used).

\begin{figure}[H]
\scriptsize
\begin{grammar}

<exp>		  ::= 	 \{ ''\%case'' : <aty> , ''exp'' : <exp> , ''\%of'' : <vbind>, ''alt'' : [ <alt> ]$^{+}$ \}	

<alt>		  ::= 	 \{ ''qdcon'' : <qdcon> , ''tbind'' : [ <tbind> ] , ''vbind'' : [ <vbind> ] , ''exp'' : <exp> \}
		  \alt 			 \{ ''lit'' : <lit> , ''exp'' : <exp> \}
		  \alt 			 \{ ''\%\_'' : <exp> \}	
\end{grammar}

\label{caseexpr}
\caption{Excerpt from appendix \ref{jscoregrammar} defining the sytax of case expressions}
\end{figure}

The expression following the "exp" keyword in the case-definition is the scrutiny. This
is usually just a variable declared in a lambda expression at a higher level.
The scrutiny ("exp") is to be bound to the variable following the "\%of" keyword.
This is simply done by replacing all instances of the variable following the "\%of" keyword
by the scrutiny.

The constructor alternative is used both as an ordinary case alternative, but it is
also used to deconstruct constructors. Since a Haskell Int is a data
constructor with name "I\#" and a value of type "Int\#", one case expression is used 
to deconstruct the "Int" value and one is used to match on the actual "Int\#" literals.

Following is a piece of JSCore pseudo-code that does this. 
The names "type1", "type2", "type3", "type4", "type5" refers to JSCore types. 
The names "exp1", "exp2", "exp3" refer to JSCore expressions. And "lit1", "lit2" refer
to JSCore literals.

\begin{figure}[H]
\lstset{ %
language=Haskell,
caption=JSCore case expression,
label=lst:int1
}
\begin{lstlisting}
{"%case": type1,
 "exp": {"qvar": "var1"},
 "%of": {"var": "var2", "ty": type2 },
 "alt": [ {"qdcon": constr1,
           "tbind": [],
           "vbind": [{"var": "var3", "ty": type3 }],
           "exp": {"%case": type4 },
                   "exp": {"qvar": "var3"},
                   "%of": {"var": "var4", "ty": type5},
                   "alt": [{"%_": exp1}
                           {"lit": lit1, "exp": exp2) }
                           {"lit": lit2, "exp": exp3) }}]}]}



\end{lstlisting}
\end{figure}

This JSCore code is rewritten to something like the following:

\begin{figure}[H]
\lstset{ %
language=Haskell,
caption=JSCore case expression in Core',
label=lst:int1
}
\begin{lstlisting}
func("case", [
    rule( const( "constr1", var3 ):stack ,
        func("case", [  
            rule( lit1:stack:var3 ,  ToCore'(stack:var3, exp1))
            rule( lit2:stack:var3 ,  ToCore'(stack:var3, exp2))
            rule( var(_):stack:var3 , ToCore'(stack:var3, exp3))
        ])
    )
])
\end{lstlisting}
\end{figure}

Note that the literals in the case-alternatives are added to the bottom of the stack,
the reason for this is that they are matched against, and the scrutiny is the first
argument in the function.
Also note that "var1" has allready
been appended to the stack, most likely by a lambda abstraction. However, 
the constructor
alternative declares a new variable that must be appended on the stack.
The JSCore types have no place in the resulting Core'.

\subsection*{Other constructs}

\subsubsection*{Module}

The "\%module" construct directly corresponds to Haskell modules. The module identifier
contains the necessary information of what package and module it belongs to, as
well as it's name.

\subsubsection*{Type definitions}

Nothing is done when visiting an algebraic data type definition, rather, it is done when 
visiting its children. (A constructor is created for each constructor definition).
% What about the type from the ADT ? and newtype ???

\subsubsection*{Constructor definitions}

When we find a constructor definition, we make a constructor. "qdcon" is the 
constructors name, and the repeating "aty" values are turned into Core' variables.

\subsubsection*{Value definitions}

When encountering a recursive value definition, we simply remember that it is
recursive by adding "True" to a stack. A stack is needed, since Value definitions
can be non-recursive within a let statement inside the recursive definition. If
a value-definition is non-recursive, we add "False" to the stack.

When we return from creating the value definitions we pop from the stack to get
back our initial "recursion" value.

The actual value definition gets a function from visiting the Expression and adds 
it to a dictionary with the name of "qvar".

\subsubsection*{Atomic Expressions}

When encountering an Atomic Expression in the form of a variable or data constructor, 
we simply look them up in the module and variable/data constructor dictionaries.

When encountering a literal, we create a new literal value, and when encountering an
Expression, we visit and return the expression.

\subsubsection*{Expression}

When encountering an Atomic Expression without arguments, we simply visit and return
it. When finding an Atomic Expression with arguments, we make a PartialApp object.
A PartialApp object is simply a temporary object that contains the Expression and the
argument. If the arity of the Expression equals the number of arguments it is turned
into a normal Core' function application.

The lambda abstraction is simply turned into a function.

The let statement simply stores the current recursion variable, and restores it after
visiting its value definition. Note that the let statement is not yet properly tested.

The case statement is the most complicated of all the statements in the Core language.
When the case statement is parsed, it is turned into a Function, the scrutiny (exp) 
is in turn applied to the function. Note that, since the only variables that are in
scope in a functions rules, is the pattern it matches against, we need to pass all 
variables downwards from the top Expression.

The note, external c-call, dynamic external c-call and label statements are currently
not implemented or tested. However, some functionality for implementing them have been
created, including a function turning a arrow-type Expression into a low-level list
of types for interfacing with C.

\subsubsection*{Argument}

The type arguments have no operational effect, and are ignored by our parser.
The Atomic Expression simply finds the referenced expression and returns it.

\subsubsection*{Case alternative}

The case alternatives all return a pair of pattern-list and Expression. The
data constructor alternative returns a list with a single constructor element and
the Expression. The literal alternative simply returns a Value object and the 
Expression it matches against. The default alternative returns a newly created
Variable and the Expression it matches for.

\subsubsection*{Binder}

The type binder has no operational effect and is ignored by our parser. The 
value binder simply visits its child and returns it.

\subsubsection*{Literal}

The literal returns a Core' Value.


\subsubsection*{Value binder}

The value binder simply returns a Core' Variable, as Core' does not have Types.


\subsubsection*{Type binder}

The type binder is simply ignored.

\subsubsection*{The others}

External-Core contains a set of other constructs. These are;
Atomic type, Basic-type, Type, Atomic-kind and Kind. These construct
have no operational effect, and are not tested.

\begin{comment}
\subsubsection*{Atomic type}

Types have no effect.

\subsubsection*{Basic type}

Types have no effect.

\subsubsection*{Type}

Types have no effect.

\subsubsection*{Atomic kind}

Kinds have no effect.

\subsubsection*{Kind}

Kinds have no effect.
\end{comment}

