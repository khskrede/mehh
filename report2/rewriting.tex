



\chapter{From Core to Core'}

The translation from Core to Core' can be seen as a graph rewriting problem. For
every construct in the Core language, there must be a rewrite rule to translate it
to the Core' language. In order to discuss these rules, we first describe a grammar
and semantics for the Core' and Core languages.

\section*{A definition of Core'}

The Core' interpreter contains the Function, Application and Constructor constructs.

A value is simply a value, an integer, a char and so on. Something that can be
operated on by functions.

A function is defined by a name and a list of pattern and function pairs,
but can also be a primitive function.
The evaluation of a function is performed by matching the arguments to
the patterns, and applying the expression it is paired with.

An application is represented by a function and an argumentlist. It is
evaluated by applying the arguments to the function.

A constructor is a function returning a Constructor object when applied. 

\section*{A Core' grammar}

We define a grammar for Core', note that the grammar is not for syntactic
purposes, but rather for rewriting rules to be defined later.

\begin{longtable}{ l c l }

\emph{application} 	& $\rightarrow$ & \emph{function} \emph{argumentlist} \\

\emph{function} 	& $\rightarrow$ & \emph{prim-function} \\
			& $|$ 		& \emph{name} \emph{patternlist} \\
			& $|$ 		& \emph{name} \emph{value} \\

\emph{patternlist} 	& $\rightarrow$ & (\emph{pattern}, \emph{function}), 
\emph{patternlist} \\
			& $|$ 		& $\epsilon$ \\

\emph{pattern} 		& $\rightarrow$ & \emph{constructor} \emph{function} ??? \\
			& $|$ 		& \emph{constructor} \emph{value} ??? \\

\emph{constructor} 	& $\rightarrow$ & \emph{name}, \emph{argumentlist ???} \\

\end{longtable}

In addition to these constructs, the Core' interpreter has now been 
extended with the following constructs:

\begin{longtable}{ l c l }

\emph{coremod} 		& $\rightarrow$ & \emph{name} \emph{typelist} \emph{valuelist}\\

\emph{typelist}		& $\rightarrow$ & \emph{type}, \emph{typelist} \\
			& $|$		& $\epsilon$ \\

\emph{parapp} 		& $\rightarrow$ & \emph{parapp} \emph{argumentlist} \\
			& $|$ 		& \emph{application} \\

\end{longtable}

Although we use the name partial application here, it is actually used to
uncurry curried functions.

These additions are for convenience, and are only usefull during 
translation.





\section*{A definition of JSCore}



\section*{Translating from JSCore to Core'}

