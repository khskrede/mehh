
\chapter{The Core language}

The Haskell 98 report mentions a Haskell kernel language, although it is not
formally defined, it is mentioned how some Haskell constructs can be translated into
simpler ones. In the case of GHC, the Core language is likely to be this simpler 
representation.

The Core language is chosen to be as small as possible, while still maintaining 
the full expressive power of Haskell.

\section{System $F_w$}

System $F$, or the second-order polymorphically typed lambda calculus...

System $F_w$ is an extension to the ... lambda calculus ... 
(System $F$ but with an arbitrarily lifted kind system... i.e. ( * $->$ ( * $->$ ( * $->$ * ) ) ))


\section{Semantics}

"Program meaning can be defined in several different ways. A straightforward intui-
tive idea of program meaning is “whatever happens in a (real or model) computer when
the program is executed.” A precise characterization of this idea is called operational
semantics."



\section{Syntax and Semantics}



The syntax described here is that of External-Core. The goal of this section is
however not so much to descirbe the syntax, but rather to describe the semantics and 
how the various constructs are related.

\subsection{Module}

The module construct is represented by the \emph{\%module} keyword, an identifier,
a list of \emph{type definitions} and a list of \emph{value definitions}:

\begin{grammar}
<module> ::= \%module <mident> \{ <tdefg> ; \} \{ <vdefg> ; \}
\end{grammar}

The module construct directly corresponds to Haskell modules. The module identifier
contains the necessary information of what package and module it belongs to, as
well as it's name.

\subsection{Type definitions}

\begin{grammar}
<tdefg> ::= \%data <qtycon> <tbind> = <cdef>
       \alt \%newtype <qtycon> <qtycon> <tbind> = <ty>
\end{grammar}


\subsection{Constructor definitions}

\begin{grammar}
<cdef> ::= <qdcon> \{ @ <tbind> \} \{ <aty> \}
\end{grammar}

\subsection{Value definitions}

\begin{grammar}
<vdefg> ::= \%rec \{ <vdef> \{ ; <vdef> \} \}
       \alt <vdef>
\end{grammar}

\begin{grammar}
<vdef> ::= <qvar> :: <ty> = <exp>
\end{grammar}

\subsection{Atomic expressions}

\begin{grammar}
<aexp> ::= <qvar>
      \alt <qdcon>
      \alt <lit>
      \alt ( <exp> )
\end{grammar}

\subsection{Expression}

\begin{grammar}
<exp> ::= <aexp>
     \alt <aexp> \{ <arg> \}
     \alt \textbackslash \{ <binder> \} -$>$ <exp>
     \alt \%let 
     \alt \%case
     \alt \%cast
     \alt \%note
     \alt \%external ccall " \{ <char> " \} <aty>
     \alt \%dynexternal ccall <aty>
     \alt \%label \{ <char> \}
\end{grammar}

\subsection{Argument}

\begin{grammar}
<arg> ::= @ <aty>
     \alt <aexp>
\end{grammar}

\subsection{Case alternative}

\begin{grammar}
<alt> ::= <qdcon> \{ @ <tbind> \} \{ <vbind> \} -$>$ <exp>
     \alt <lit> -$>$ <exp>
     \alt \%\_ -$>$ <exp>
\end{grammar}

\subsection{Binder}

\begin{grammar}
<binder> ::= @ <tbind>
        \alt <vbind>
\end{grammar}

\subsection{Type binder}

\begin{grammar}
<tbind> ::= <tyvar>
       \alt ( tyvar :: <kind> )
\end{grammar}

\subsection{Value binder}

\begin{grammar}
<vbind> ::= ( <var> :: <ty> )
\end{grammar}

\subsection{Literal}

\begin{grammar}
<lit> ::= ( [-] \{ <digit> \} :: <ty> )
     \alt ( [-] \{ <digit> \} \% \{ <digit> \} :: <ty> )
     \alt ( <char> :: <ty> )
     \alt ( \{ <char> \} )
\end{grammar}

\subsection{Atomic type}

\begin{grammar}
<aty> ::= <tyvar>
     \alt <qtycon>
     \alt ( <ty> )
\end{grammar}

\subsection{Basic type}

\begin{grammar}
<bty> ::= <aty>
     \alt <bty> <aty>
     \alt \%trans <aty> <aty>
     \alt \%sym <aty>
     \alt \%unsafe <aty> <aty>
     \alt \%left <aty>
     \alt \%right <aty>
     \alt \%inst <aty> <aty>
\end{grammar}

\subsection{Type}

\begin{grammar}
<ty> ::= <bty>
    \alt \%forall \{ <tbind> \} . <ty>
    \alt <bty> -$>$ <ty>
\end{grammar}

\subsection{Atomic kind}

\begin{grammar}
<akind> ::= *
       \alt \#
       \alt ?
       \alt <bty> :=: <bty>
       \alt ( <kind> )
\end{grammar}

\subsection{Kind}

\begin{grammar}
<kind> ::= <akind>
      \alt <akind> -$>$ <kind>
\end{grammar}





\section{Translation into the Stg language}

% Understanding this may help us understand how to write
% our translation.


\section{GHC Extensions to Haskell 98}

\subsection{Existensial quantification}
