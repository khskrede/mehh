
\chapter{Conclusion and Future Work}
\label{chap:conc}


% Write some stuffs here

\section{Serializing and deserializing}
\label{serdes}

The current situation regarding the serialization of Haskell programs into 
the JSCore format have several problems. It is dependent on a buggy part of GHC, 
and on a poorly maintained package (extcore). The result of these bugs is that
GHC exits with a panic-error when compiling some Haskell modules. And this in
turn, means that we have been unable to create JSCore for the GHC boot libraries.

Currently, the libraries necessary to run some simple Haskell programs have been
implemented at a high level in Python. This implementation is not "correctly"
implemented with regard to the Haskell language, but it is sufficient in order
to test some simple programs.

Some alternative methods have been investigated as solutions for this problem,
including the use of the Cabal (Haskell package system) API in collaboration
with the GHC API in order to create the JSCore files. However, Cabal interface
with GHC through command line arguments. 
It was also attempted to create the necessary functionality using just the GHC
API, however this also turned out to be problematic. Though it now seems like
this is the best way to move forward. By using the main module as the base for
the serializer, it should be possible to interface with the build-system
exactly like GHC does. This would mean that it would be possible to write a
fairly simple Haskell function to dump the intermediate format used by GHC
to JSCore, and since this program would function exactly like GHC, anything 
would be possible to compile using it.

The deserialization should not change much, as the JSCore intermediate format
would not have to change very much. If anything, the JSCore format can be
simplified, and made to match better against Core and System $F_C$ than
External-Core does. It is however, necessary to improve the serialization
functionality before the deserialization can be improved much as the programs
that can be serialized is of little complexity, and therefore does not test
very much.


\section{Benchmarking and testing}

\begin{comment}
Some simple benchmarking and testing was done. The benchmarking was done by
running the naive fibonacci program with $n=40$, the results can be seen in
table \ref{tab:benchmarks}.
It is clear that the interpreter must be optimized 
further if it is to compete with GHC. A significant increase in execution 
time with the JIT as opposed to without the JIT was observed. In addition,
for this benchmark, the interpreter with JIT was approximately $3.3$ times faster than
$runhaskell$ (an interpreter that comes with GHC).
\end{comment}

The preliminary benchmarking that was done was described in chapter \ref{chap:bench}. 
The results
show a significant decrease in the execution time with the JIT compared to without it.
The interpreter also outperforms the "runhaskell" interpreter that comes with GHC. It
is however slower than GHC, though that is to be expected, as a lot of 
development and optimization is left.

The testing performed has been described in chapter \ref{chap:impl}. The results of the 
testing are that the current implementation is sufficient for simple programs taking 
advantage of various Haskell constructs. It is also clear that this is not sufficient
to run any more complicated programs, but it serves well for the future development of
the system. Eventually, the interpreter should run the GHC benchmarking suite 
(NoFib \cite{partain1992nofib}) and the GHC testsuit.

\section{Concluding remarks}

%NoFib GHC benchmark

The goal of this project was to investigate whether Haskell could benefit from JIT 
compilation, specifically; the techniques applied by the PyPy project. 
To answer this, it was attempted to implement a full Haskell compilation
system, using an interpreter called Haskell-Python\cite{haskellpython} as the back-end,
and GHC as the front-end, and translate it into a JIT compiler using the PyPy translation
toolchain. The work involved in this project has mostly been the implementation of the necessary
tools to get Haskell programs from GHC into Haskell-Python. This has only been partially
successful, as discussed in section \ref{serdes}.

Unfortunately, since so much time went into trying to get GHC to cooperate, 
the original question remains unanswered at this stage. This is due to the fact that the
approach taken to test the hypothesis has not been successfully completed. However,
based on the results from this partially implemented system, its continued
development may result in a fast Haskell compilation system, and may eventually prove
the hypothesis true.

