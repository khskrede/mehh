
\chapter{Conclusion and Future Work}
\label{chap:conc}

This is the final chapter of the report, it briefly discusses the future work that 
is to be done, and a conclusion.

\section{Serializing and deserializing}
\label{serdes}

The current situation regarding the serialization of Haskell programs into 
the JSCore format have several problems. It is dependent on a buggy part of GHC, 
and on a poorly maintained package (extcore). The result of these bugs is that
GHC exits with a panic-error when compiling some Haskell modules. This
means that we have been unable to create JSCore for all the GHC boot libraries.

Currently, the libraries necessary to run some simple Haskell programs have been
implemented at a high level in Python. This implementation is not "correctly"
implemented with regard to the Haskell language, but it is sufficient in order
to test some simple programs.

Some alternative methods have been investigated as solutions for this problem,
including the use of the Cabal (Haskell package system) API in collaboration
with the GHC API in order to create the JSCore files. However, Cabal interacts
with GHC through command line arguments. 
It was also attempted to create the necessary functionality using just the GHC
API, however this also turned out to be problematic due to the limited Haskell 
experience and the overall complexity of GHC. Though it now seems like
using the GHC API is the best way to move forward. By using the main module of GHC
as the base for
the serializer, it should be possible to interface with the build-system
exactly like GHC does. This would mean that it would be possible to write a
fairly simple Haskell function to dump the intermediate format used by GHC
to JSCore, and since this program would function exactly like GHC, anything 
would be possible to compile using it.

The deserialization should not change much, as the JSCore intermediate format
would not have to change much. If anything, the JSCore format can be
simplified, and made to match better against Core and System $F_C$ than
External-Core does. It is however necessary to improve the serialization
functionality before the deserialization can be improved. As the programs
that can be serialized is of little complexity, and the deserializer can only be
tested by these programs.


\section{Benchmarking and testing}

Some preliminary benchmarking was done and described in chapter \ref{chap:bench}. 
The results
show a significant decrease in the execution time with the JIT compared to without it.
The interpreter also outperforms the "runhaskell" interpreter that comes with GHC. It
is however slower than GHC, though that is to be expected as the interpreter is still 
at an early stage of development.

The testing performed has been described in chapter \ref{chap:impl}. The results of the 
testing are that the current implementation is sufficient for simple programs taking 
advantage of various Haskell constructs. It is also clear that this is not sufficient
to run any more complicated programs, but it serves well for the future development of
the system. Eventually, the interpreter should run the GHC benchmarking suite 
(NoFib \cite{partain1992nofib}) and the GHC testsuit.

\section{Concluding remarks}

%NoFib GHC benchmark

The goal of this project was to investigate whether Haskell could benefit from JIT 
compilation, specifically the techniques applied by the PyPy project. 
To answer this, it was attempted to implement a full Haskell compilation
system, using an interpreter called Haskell-Python\cite{haskellpython} as the back-end,
and GHC as the front-end, and translate it into a JIT compiler using the PyPy translation
toolchain. The work involved in this project has mostly been the implementation of the necessary
tools to get Haskell programs from GHC into Haskell-Python. This has only been partially
successful, as discussed in section \ref{serdes}. 

Since so much time went into trying to get GHC to cooperate, this thesis work never reached the
maturity that was aimed for. The consequence of this is that the original hypothesis remains
unanswered. It cannot be said that Haskell can benefit from the JIT approach as opposed to a
statically optimizing approach. It can be said however, that the interpreter implemented here 
clearly benefits from the JIT compiler, and that future development may prove the hypothesis true.

