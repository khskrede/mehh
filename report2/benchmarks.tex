
\chapter{Preliminary benchmarks}
\label{chap:bench}

Since the interpreter can be successfully translated by the RPython toolchain, 
it is possible to do some simple preliminary benchmarking. This chapter briefly 
discusses how the benchmarking was done and the results it gave.

\section{Specifications}

The benchmarking was done on a Compal NBLB2 laptop, see relevant specifications
in table \ref{tab:specs}

\begin{table}[H]
\centering
\begin{tabular}{l|l}
\hline
\hline
Processor   & Intel Core i7-640M (2x2.8GHz) \\
Memory      & 2x2GB DDR3-1333 \\
Chipset     & Intel PM55 \\
OS          & Arch Linux (3.3.7-1-ARCH) \\
GHC version & 7.4.1 \\
\hline
\end{tabular}
\caption{Relevant system specifications}
\label{tab:specs}
\end{table}

\section{Translation}

The translation of the interpreter was done with the current head of the 
PyPy repository (branch: default, revision 04e2b329ede5). The version of the interpreter that has
been benchmarked are from the current head of the Haskell-Python repository (branch: khs,
revision: 53b7f6b71cd1).

Translating to C is done by running the PyPy translator tool with the path to the
file defining the target as argument:

\begin{figure}[H]
\lstset{ %
language=bash,
}
\begin{lstlisting}
$ pypy <pypy-repo>/pypy/translator/goal/translate.py <pyhaskell-repo>/targethaskellstandalone.py
\end{lstlisting}

\end{figure}

Translating with the JIT is done by feeding the translator tool with the argument "--opt=JIT":

\begin{figure}[H]
\lstset{ %
language=bash,
}
\begin{lstlisting}
$ pypy <pypy-repo>/pypy/translator/goal/translate.py --opt=JIT <pyhaskell-repo>/targethaskellstandalone.py
\end{lstlisting}

\end{figure}




\section{Test program}

The application used to do the benchmarking was a naive implementation of the 
fibonacci program, see listing \ref{lst:naivfib}.

\begin{figure}[H]

\lstset{ %
language=Python,
caption=Naive fibonacci implementation in Haskell,
label=lst:naivfib
}
\begin{lstlisting}
main = do 
    let n = 40
    putStrLn $ "Fib(" ++ (show n) 
               ++ ") = " ++ (show $ fib n)

fib :: Int -> Int
fib 0 = 0
fib 1 = 1
fib n = fib (n-1) + fib (n-2)
\end{lstlisting}

\end{figure}

\section{Results}

The benchmarking was done by running the naive fibonacci program with $n=40$. 
The results can be seen in table 
\ref{tab:benchmarks}. 

\begin{table}[H]

\centering
\begin{tabular}{l|l}
\hline
\hline
Compilation System & Time \\
\hline
GHC               & 0m 11.296s  \\
runhaskell        & 4m 23.803s \\ 
PyHaskell no JIT  & 12m 57.254s \\
PyHaskell JIT     & 1m 20.638s \\
\hline
\end{tabular}

\caption{Timing results for running the naive fibonacci implementation with n=40. Results
are from the "user" output of the "time" linux command.}
\label{tab:benchmarks}

\end{table}

From these results it is clear that the interpreter must be optimized further if
it is to compete with GHC. A significant decrease in execution time with the JIT
as opposed to without the JIT was observed. In addition, for this benchmark, the
interpreter with JIT was approximately $3.3$ times faster than $runhaskell$ 
(an interpreter that comes with GHC).



