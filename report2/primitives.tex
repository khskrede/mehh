

\chapter{Primitives and Libraries}
\label{chap:prims}

GHC uses a module called GHC.Prim, located in the ghc-prim package. This
module is generated automatically from a file called "primops.txt.pp", this file
contains information about the machine representations of the various types, as
well as the operations that can be performed on them.

\section{The initial plan}

Initially, it was assumed that it would be trivial to translate the Haskell
libraries used by GHC into Core. This way, it would only be necessary to implement
the primitive types and functions.

Unfortunately, this turned out not to be the case. Due to bugs in GHC, it was not 
possible to generate Core without changing GHC code, and this would be a daunting task.
Some alternatives was considered;

\section{Our approach}

Why did we choose to go with External-Core ???

...

\section{Why it does not work}

Why External-Core did not work for us...

...

\section{How to fix it}

... Use the GHC API ... Figure out how to compile multiple files with
the API, how to set language extensions, how to. Using cabal with the API.. 
Poorly documented...


\section{Our current implementation}

How we cheat...
