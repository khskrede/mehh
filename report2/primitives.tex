

\chapter{Primitives and Libraries}


\begin{quotation}
"Over the fifteen years of its life so far, GHC has grown a huge num-
ber of features. It supports dozens of language extensions (notably
in the type system), an interactive read/eval/print interface (GHCi),
concurrency (Peyton Jones et al., 1996; Marlow et al., 2004), trans-
actional memory (Harris et al., 2005), Template Haskell (Sheard
and Peyton Jones, 2002), support for packages, and much more be-
sides. This makes GHC a dauntingly complex beast to understand
and modify and, mainly for that reason, development of the core
GHC functionality remains with Peyton Jones and Simon Marlow,
who both moved to Microsoft Research in 1997." - (A History of Haskell: Being lazy pp.29)
\end{quotation}


GHC uses a module called GHC.Prim, located in the ghc-prim package. This
module is generated automatically from a file called "primops.txt.pp", this file
contains information about the machine representations of the various types, as
well as the operations that can be performed on them.

\section{GHC Primitives}

\subsection{Foreign functions}

\subsubsection{C functions}

\subsubsection{C-- functions}

\section{GHC Libraries}



\section{GHC Prelude}

Prelude lives in Base.

The GHC Prelude is a module that is implicitly imported into every Haskell program.

