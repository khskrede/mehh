

\section{Primitives and Libraries}
\label{chap:prims}

% TODO: Present all library functionaliyt implemented !

Following is a description of some of the most notable implemented library 
functions.

All the builtin functionality is located in the "toplevel/pyhaskell/builtin/"
folder. When describing the implementation of a Value or a Function, we note
the Python file and the Haskell module it refers to.

\subsection{Primitive Values}

A few primitive Value types has been implemented. The ones that are currently
tested are Char and Int. See listing \ref{lst:int} for an example of a 
Value implementation. 

% TODO: _immutable_fields_ = ["value"] hit to the translator

\begin{figure}[H]
\lstset{ %
language=Python,
caption=Python class implementing the Haskell Int Value.,
label=lst:int
}
\begin{lstlisting}
class Int(haskell.Value):
    _immutable_fields_ = ["value"]

    def __init__(self, integer):
        assert isinstance(integer, int)
        self.value = integer

    def match(self, other, subst):
        value = other.getvalue()
        if value:
            assert isinstance(value, Int)
            if self.value == value.value:
                return haskell.DEFINITE_MATCH
            return haskell.NO_MATCH
        return haskell.NEEDS_HNF

    def __eq__(self, other):
        return (isinstance(other, Int) and self.value == other.value)

    def __ne__(self, other):
        return not (self == other)

    def tostr(self):
        return str(self.value)
\end{lstlisting}
\end{figure}

\subsection{Lists}

The cons (or, ":") constructor. A constructor is created with the "haskell.constr()"
function.

A Python-decorator is used to turn a Python function into a Haskell primitive function.
The decorator "@mod.expose(name, arity)" implemented in the Module class, turns a Python 
function into a haskell function by creating a PrimFunction and adding it to the 
variable-dictionary of the "mod" module. See \ref{lst:cons} for an example.

\begin{figure}[H]
\lstset{ %
language=Python,
caption=Implementation of the Haskell ++ (concatenation) operator.,
label=lst:cons
}

\begin{lstlisting}
mod = module.CoreMod("base:GHC.Base")

@mod.expose("++", 2)
def concatenation(args):
    def conc(a, b):
        if len(a.getargs() ) > 1:
            return haskell.constr(":", a.getarg(0), conc(a.getarg(1), b) )
        else:
            return b

    a, b = args
    t = conc(a,b)

    return t 
\end{lstlisting}
\end{figure}


The end of a list refors to a "[]" constructor (haskell.constr("[]")).


\subsection{unpackCString}

The unpackCString function simply takes a 'Addr' object and returns a
Haskell String. A Haskell String as a linked list of Chars. \ref{lst:unpack}

\begin{figure}[H]
\lstset{ %
language=Python,
caption=Implementation of the unpackCString function.,
label=lst:unpack
}

\begin{lstlisting}
mod = module.CoreMod("ghc-prim:GHC.CString")

@mod.expose("unpackCString#", 1)
def unpackCString(args):
    a = args[0]
    b = str(a.value)
    t = types.zmzn 
    for i in range(len(b)-1, -1, -1):
        c = haskell.constr( "Czh", Char(b[i]) )
        t = haskell.constr( ":", c, t )
    return t 
\end{lstlisting}
\end{figure}

\subsection{putStrLn}

The putStrLn function takes a Haskell String and prints the characters in it,
followed by a 'newline'.

\begin{figure}[H]
\lstset{ %
language=Python,
caption=Implementation of the putStrLn function.,
label=lst:put
}

\begin{lstlisting}
mod = module.CoreMod("base:System.IO")

@mod.expose("putStrLn", 1)
def putStrLn(args):
    t = args[0]
    while len(t.getargs()) > 1:
        sys.stdout.write( t.getarg(0).getarg(0).value )
        t = t.getarg(1)
    sys.stdout.write("\n")
    return args[0]
\end{lstlisting}
\end{figure}


\subsection{The Num class}

The Num class implements some generic functions. The ones implemented
in our Python library is multiplication, subtraction and addition. These
are pretty much implemented in the same way, so we only present the
multiplication function as an example here, see listing \ref{lst:mul}.
Note that the function takes 3 arguments, the first argument 
(in the fibonacci test case, \$fNumInt) is used to tell the
generic multiplication function which specific function to use.
See subsection \label{subsec:classes} for a description. In our 
implementation however, the first argument is just an Int that we 
check against.

\begin{figure}[H]
\lstset{ %
language=Python,
caption=Implementation of the generic multiplication function.,
label=lst:mul
}

\begin{lstlisting}
@mod.expose("*", 3)
def multiply(args):
    ty, a, b = args
    if ty == mod.qvars["$fNumInt"]:
        return haskell.make_application(izhconstr,
            [haskell.make_application(prim.multiply, [a.getarg(0), b.getarg(0)])])
    else:
        raise NotImplementedError
\end{lstlisting}
\end{figure}


