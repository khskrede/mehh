
\documentclass{book}

\usepackage{verbatim}
\usepackage{cite}
\usepackage{parskip}
\usepackage{longtable}
\usepackage{listings}
\usepackage{graphicx}
\usepackage{float}
\usepackage{rotating}
%\usepackage{fancyhdr}
%\usepackage[margin=.7in]{geometry}
\usepackage{appendix}

\usepackage{color}
\definecolor{Grey}{rgb}{0.5,0.5,0.5} 

% Make margin of captions smaller
%\usepackage[margin=1cm]{caption}



% Fancy headers
\usepackage{fancyhdr}
\pagestyle{fancy}

% -------------------------
% Test stuff with listings
% -------------------------
\usepackage{xcolor}
\usepackage{caption}
\DeclareCaptionFormat{listing}{%
  \parbox{\textwidth}{\colorbox{Gray}{\parbox{\textwidth}{#1#2#3}}\vskip-4pt}}

\lstset{ %
basicstyle=\scriptsize,         % the size of the fonts that are used for the code
%numbers=left,                   % where to put the line-numbers
numberstyle=\scriptsize,        % the size of the fonts that are used for the line-numbers
%frame=trBL%single,                   % adds a frame around the code
frame=single,
framerule=1pt,
frameround=tttt,
%backgroundcolor=\color{red},
showstringspaces=false,
tabsize=2,                      % sets default tabsize to 2 spaces
captionpos=b,                   % sets the caption-position to bottom
breaklines=true,                % sets automatic line breaking
breakatwhitespace=false,        % sets if automatic breaks should only happen at whitespace
%title=\lstname,			% include title (file name)
xleftmargin=5pt,
xrightmargin=5pt,
%rulecolor=\color{Grey}
}

% -------------------------
% Test with section numbers in the margin
% -------------------------
%\makeatletter
%\def\@seccntformat#1{\color{Grey}\llap{\csname the#1\endcsname\quad}}
%\makeatother


% Different font in captions
\newcommand{\captionfonts}{\small}

\makeatletter  % Allow the use of @ in command names
\long\def\@makecaption#1#2{%
  \vskip\abovecaptionskip
  \sbox\@tempboxa{{\captionfonts #1: #2}}%
  \ifdim \wd\@tempboxa >\hsize
    {\captionfonts #1: #2\par}
  \else
    \hbox to\hsize{\hfil\box\@tempboxa\hfil}%
  \fi
  \vskip\belowcaptionskip}
\makeatother   % Cancel the effect of \makeatletter

% Options for footnotes
\setlength{\footnotesep}{0.5cm}
%\setlength{\skip\footins}{2cm}

\begin{document}

%\pagestyle{fancy}


\title{Just-In-Time compilation of\\Haskell using PyPy and GHC}
\author{Knut Halvor Skrede}

% Frontpage
\begin{titlepage}
\maketitle

%\begin{abstract}
%The paper describes a system for using GHC as frontend and PyPy as backend for a 
%Haskell JIT compiler. An intermediate language in JSON format based on GHC's Core
%language is described. The implemented parts are a serializer written in Haskell and
%a deserializer written in Python, in addition to some Haskell library functions.
%The project is meant to serve as a base for further development. 
%\end{abstract}

\end{titlepage}


\tableofcontents

\clearpage

\listoffigures
\listoftables

\clearpage

\setlength\LTleft{0pt}
\setlength\LTright{0pt}


\chapter{Introduction}

This chapter discusses the motivation behind the project and 
presents a description of the project and of the work done. It 
is concluded by introducing the remaining chapters.

\section{Motivation and project description}

The project aims to investigate the feasibility of JIT (Just-in-time) 
compilation of a strongly-typed purely-functional language. Since
programs written in such a language can be heavily optimized at 
compile time, it is uncertain whether such programs can benefit from
JIT compilation. However, a JIT compiler has a lot more information to
work with than a static compiler. 

To test this, the techniques 
used by the PyPy (Python in Python) project are applied to the Haskell 
programming language. By implementing the back-end for a Haskell compiler 
in RPython (restricted Python), and using GHC (the Glasgow Haskell Compiler) 
as a front-end, a full Haskell JIT compiler can be implemented. An 
interpreter for a language very similar to the intermediate language used
by GHC was already implemented, and this is the base for this project.
Although it is already possible to have JIT compilation with GHC through
its LLVM back-end, the PyPy approach will interpret code at a much higher level.

We focus on implementing a serializer from Haskell to an intermediate
format using GHC, and a deserializer from that format to the interpreter. The 
interpreter is for a language similar to Core (the intermediate format used by GHC.)
By implementing some simple programs in Haskell, and running them through our 
compilation system, 
we hope to show that the methods of the PyPy project can be successfully applied 
to pure functional languages such as Haskell.

\section{Contributions}
The contributions of this thesis is a description of Haskell-Python (an
interpreter for Core' (a lambda-calculus inspired by Haskell)), and a system for
translating Haskell programs into Core'. In addition to this, the paper 
presents a description of the full compilation system in its current state,
and a plan for the future development of the system based on the 
successes and failures so far.

Following is a listing describing how the work in this project has been
partitioned.

The following work had already been done:
\begin{itemize}

\item The Haskell-Python\cite{haskellpython} interpreter was written.

\end{itemize}

These parts were the result of an earlier stage of the project:
\begin{itemize}

\item An investigation into the use of GHC as a front-end for the compiler.
This resulted in the JSCore intermediate language, but was unsuccessful
at creating JSCore files from the GHC Haskell libraries.

\item A parser for the JSCore language was implemented; however, this 
parser was only successful for a very small subset of the JSCore language.

\item A simple system for testing functionality was implemented.

\end{itemize}

For this project, the following has been done:
\begin{itemize}

\item Another attempt was given at the creation of JSCore for the GHC 
Haskell libraries, but was still unsuccessful. The reason for this being
bugs in GHC. Specifically, bugs in the code that dumps the External-Core
files.

\item Various Haskell values and functions have been implemented at a higher
level. Due to the fact that we could not successfully create intermediate
files for the Haskell
libraries we had to implement the functionality at a higher level, i.e. 
Python.

\item Based on the gained understanding of the Core language from the
previously mentioned work, the parser has been improved. The result of 
the improvements to the parser, with the implementation of the libraries,
has resulted in the successful execution of some less trivial programs,
such as a naive recursive implementation of fibonacci.

\end{itemize}

In addition to this, some work has been done in parallell by E.W.Thomassen,
most notably:
\begin{itemize}

\item Refactoring of the code written previously, in order to make it better 
match other PyPy projects.

\item Rewriting the Python code into RPython, such that it can be translated by
the PyPy toolchain.

\item He has also changed quite a bit of the functionality for the better,
and details of this work can be found in his report.

\end{itemize}

% Write about the use-cases of laguages like Haskell, and the benefits of 
% Jit compilation

% + Research blahblah... Just to see if it works well.

% Structure of the paper. TODO
\section{Structure of the paper}
In chapter \ref{chap:back} some background information is presented, including 
some terms and concepts. 
%
A description of the 
Haskell-Python interpreter is given in chapter \ref{chap:hs}.
%
Chapter \ref{chap:rewrite} goes into detail regarding the intermediate 
languages, and the mapping between them.
%
%In chapter \ref{chap:prims} the Haskell libraries and necessary primitives
%are discussed.
%
%Chapter \ref{chap:pipe} describes the entire pipeline of the compilation system 
%at a high level.
%
%Chapter \ref{chap:test} describes the test-system used.
%
Chapter \ref{chap:impl} contains a description of the compilation system.
%
In chapter \ref{chap:similar} a brief discussion of similar work is presented.
%
Chapter \ref{chap:bench} presents some preliminary benchmark results.
%
Everything is concluded in chapter \ref{chap:conc}, which discusses 
the results and future work.



\section{Background}
\label{chap:back}

\subsection{Haskell}

% What is Haskell ?
Haskell is a lazy, pure functional language with non-strict semantics and static 
polymorphic typing. It provides user-defined algebraic datatypes, pattern-matching, 
list comprehensions, a system for modules, a system for monadic I/O, and a large 
standard library. In addition to beeing strongly typed, Haskell supports type
inference, which means type annotations are seldom required. We do not go into 
too much detail here, for a full description 
of Haskell, see "the Haskell 2010 language report"\cite{marlow2010haskell}. 
Or for a complete history of Haskell, see "A history of Haskell: being lazy 
with class"\cite{hudak2007history}.

Two specific features
of Haskell stand out: it is purely functional; this means that the functions 
can not have side-effects, or mutate data. For equal arguments, a function 
must provide equal results. The fact that Haskell is lazy refers to the techniques
used to evaluate a Haskell program, meaning that the arguments to a function are passed
unevaluated, and only evaluated when needed. Lazy semantics also means that impure 
non-functional language features are impossible, as the two cannot work in conjunction.
\cite{marlow2010haskell, marlow2012glasgow}

\subsection{PyPy}

% What is the PyPy project ?
PyPy is a project that shows the feasability of constructing a VM (Virtual Machine) 
for a dynamic  language in a dynamic language, specifically, Python. The PyPy 
environment aims to translate (i.e. compile) the VM into arbitrary targets. This 
means that an interpreter constructed in the PyPy environment will be able to 
run on any target supported by PyPy. Instead of writing multiple versions of 
the interpreter (one for C/Posix, JAVA, and one for CLI/.NET), it can be 
written in RPython, and translated to those back-ends. PyPy uses the 
"meta-programming" argument, if the VM can be written at a level of abstraction 
high enough, then it can be translated to any lower level platform. Implementing 
programming languages using a direct encoding approach is a complex task, and it 
usually results in an implementation that is specifically designed for a target platform. 
In effect this means that the implementation is not generic, and it is difficult to 
reuse the code for anything other than it's specific purpose. In contrast, 
PyPy's approach puts weight on portability and reusability\cite{pypy}. Although the
PyPy project puts most of its effort into its Python implementation, other projects
(such as a Prolog implementation \cite{bolz2010towards}) clearly shows the benefits of
such a generic approach.

% What is RPython and why is it used ?
The methods used by PyPy is to implement interpreters in RPython. RPython is a proper 
subset of Python (a RPython program can be executed by a Python interpreter) 
chosen such that it is possible to perform type inference on it. This
means that RPython programs can be translated into efficient C programs. Translating a 
program to C adds a number of implementation details that are not present in the RPython
implementation, such as a garbage collector. In addition to this, a tracing-JIT compiler 
can be added semi-automatically. This means that writing an interpreter in RPython containing
a tracing-JIT is much easier and less error-prone than implementing a specialized tracing-JIT
would be. 
\cite{bolz2011runtime}

% What is a JIT? TODO ???
PyPys tracing JIT is used to trace the execution of a number of languages implemented 
in RPython. In effect,
this JIT works on the meta-level, tracing the execution of the interpreter, and not the 
execution of the program being interpreted. The approach is called meta-tracing. In addition
to just tracing the meta-level, the RPython translator (the Python program that translates
RPython to C) allows for some annotations, speeding up the JIT. The efficiency of the 
resulting dynamic compiler relies on information fetched during runtime. Slowly changing 
variables is an example of such information, as it can be exploited by the compiler at 
run time by compiling multiple instances of code (one for each value of the variable),
resulting in faster code. \cite{bolz2011runtime}

% What is a tracing JIT ?
A tracing-JIT works by recording the execution of a program. The result is a set of 
traces of concrete execution, these traces are linear lists of operations. The lists of
operations are then optimized and turned into machine code. Among other benefits, the
result is free inlining of functions, as the functions operations are simply added to
the trace. \cite{bolz2011runtime}

\subsection{Haskell-Python}

% What is Haskell-Python ?
Haskell-Python\cite{haskellpython}
is an interpreter for a Haskell inspired lambda-calculs called Core' (Core marked).
It is meant to serve as the back-end for a complete Haskell compiler, after taking advantage 
of the front-end abilities of GHC. The interpreter notably has support for pattern matching 
and constructors. Our intent is to extend this interpreter into a full Haskell interpreter.
For more on Haskell-Python, see subsection \ref{chap:hs}.

% What is lambda-calculus ??? TODO



\subsection{GHC}

GHC is a 20 years old project, and it has been under development during
all these years. It started out with the goals of beeing a freely available
robust and portable compiler for Haskell, to provide a modular framework that
could be extended and developed by other researchers, and to learn how real
Haskell programs behave. \cite{marlow2012glasgow}

GHC can be divided into three parts, the compiler, the boot libraries
(libraries the compiler itself depends on) and the RTS (Runtime System). 
The compiler is the part
that turns Haskell source code into executable code. The boot libraries are the 
libraries that the compiler itself depends on. The RTS is a large library
of C code that is responsible for running the Haskell programs, such as the 
GC (Garbage Collector) implementation. The RTS system is linked into all 
Haskell programs compiled by GHC. These three parts corresponds to subdirectries
in the GHC source, namely "compiler", "libraries" and "rts".
\cite{marlow2012glasgow}

The compiler can also be divided into three parts. The Compilation Manager is 
responsible for the compilation of multiple Haskell source files. Its job is to
determine the order in which the files must be compiled. The Haskell Compiler 
(abbreviated "Hsc" inside GHC), handles the compilation of a single Haskell source
file. The Pipeline handles any preprocessing that is necessary, and the output
from Hsc is usually an assembly file that must be fed to an assembler.
\cite{marlow2012glasgow}

% TODO! THIS IS WHERE I STOPPED!

\subsubsection{GHC API}

% TODO
The compiler is (in addition to being a binary) itself a library that exports an API.
The GHC API was a goal from the beginning of its development, in the wording of
"being modular". A few notable projects have taken advantage of this modularity,
including a version of GHC containing a Lisp front-end, and a version that generates
Java code. With the growing popularity of Haskell, interest in tools that deal with
the language has increased. These tools need a lot of the functionality that is already
present in GHC. For this reason, GHC is built as a library, rather than a monolithic
program. The library is linked by a small Main module. In addition to this, GHC
exposes an API to deal with the library.\cite{marlow2012glasgow} 

\subsubsection{The Runtime System}

The RTS provides the support that is necessary for a Haskell program to run, this
includes; memory management, scheduling and thread management, primitive operations
and a bytecode interpreter and dynamic linker for GHCi (GHCs interactive environment).
\cite{marlow2012glasgow} 

\subsubsection{Core}

% TODO ???
Haskell is intended to be easy to read and write by humans. For this reason, it
incorporates a lot of syntactic constructs. This means that there are
usually many ways to write the same program. The definition of the Haskell language
defines these syntactic constructs in terms of their translation into simpler
constructs. Many of these syntactic constructs are thus in effect syntactic sugar.
After removing all of the syntactic suger (desugaring) GHC is left with a much 
simpler language. This language is called Core (or system $F_C$ when referring to
the theory).\cite{marlow2012glasgow} We discuss Core in more detail in subsection 
\ref{chap:rewrite}.

\subsubsection{External-Core and Linkcore}

% What is External-Core ?
GHC uses an intermediate language throughout it's 
simplification phase. The External-COre project presents a formal definition of the syntax 
of this language. And in addition to this, enables the representation to be exported 
to files. The idea is that this allows compiler implementors and researchers to use GHC
as a front-end for Haskell compilers. Before outputting External-Core files,
the Haskell files are typechecked, desugared and simplified. \cite{tolmach2010ghc}

% What is linkcore ?
The linkcore project implements a linker for Core programs, i.e. it transforms
a single Haskell module into a single closed External-Core module. In addition to
this, since the linker requires External-Core representation of the ghc-libraries,
it also contains instructions on how to create these. 

% They have both bitrotted!
Unfortunately, at the time of this writing, both the External-Core functionality in
GHC, the extcore and the linkcore 
packages have bitrotted. Although there seems to be interest for the continued 
maintainance of External-Core in GHC.

\subsection{Similar work}

In addition to the Python implementation, PyPy implements a low-level 
hardware emulator (PyGirl), a PHP interpreter, and a Prolog interpreter. 
Various other experiments have also been created by the PyPy team. 

\subsubsection{PyPy Prolog}

In addition to the implementation of Python, PyPy has also shown that its techniques
are applicable to other languages. The Prolog VM is an example of this. Implementations
of Prolog are usually written in low-level languages such as C, this usually results in
good performance, but means they are difficult to write and maintain. The PyPy Prolog 
interpreter clearly outperforms other Prolog interpreters written in other high-level
languages, and it also outperforms state-of-the-art Prolog VMs at specific benchmarks,
which shows that other Prolog implementations can benefit from the techniques used by
PyPy. \cite{bolz2010towards}

% TODO !
\subsubsection{HappyJIT}

PHP (Hypertext Preprocessor) is a language used to develop the server-side of 
websites. The users request for a website is received by the server, the PHP script
then executes the request, often involving querying a database and then generating 
the actual HTML for the user. Increasing the effectiveness of this process would
reduce the time it takes for a user to have a website request answered. 
The HappyJIT project implements a PHP interpreter in RPython, this interpreter is 
translated by the PyPy translator into a tracing-JIT. The approach show that 
the techniques significantly improve the performance of several common use cases.
\cite{homescu2011happyjit}

% TODO 
\subsubsection{PyGirl}

As a case study, PyGirl implements an emulater for the Nintendo Game Boy. The project 
shows the feasibility of implementing a low-level VM for hardware in a high-level 
language to improve portability, and reduce complexity. The project shows that the
reduction in implementation complexity with this approach is substantial, 
and that the performance loss can be insignificant.
\cite{bruni2009pygirl}

% TODO !
\subsubsection{The GHC LLVM back-end}

The LLVM (Low Level Virtual Machine) is a framework for the optimization of 
programs from the compilation phase to runtime. The LLVM provides high-level information 
to the compilation system during compile-time, run-time and in idle time between
runs. By creating codegenerators for the virtual instruction set supported by
LLVM, implementors can take full advantage of its features.
\cite{lattner2004llvm}

GHC can generate LLVM code from Cmm (C minus minus; is a low-level imperative
language with an explicit stack). In some cases the 
LLVM back-end can produce significantly faster code than the traditional route. 
\cite{marlow2012glasgow, terei2010llvm}







\chapter{Haskell-Python Core' interpreter}

Haskell-Python is an interpreter for a language similar to Core, we call it Core' here.
For a class-diagram, see .....

\section{Classes}

All base classes inherit from $object$.

\subsection{Symbol}

A cached symbol that can be compared by identity (which is not true for strings).

\subsection{HaskellObject}

Base class for objects that the interpreter handles.

\subsubsection{Value}

\begin{itemize}
\item{$Constructor$} A constructor. This is an abstract base class, subclasses are 
generated in ConstructorN for various arguments.
\item{$AbstractFunction$} The $AbstractFunction$ is a base class for functions. The 
$Function$ and $PrimFunction$ classes inherits it. The $Function$ class describes
user-defined functions (i.e. written in haskell). The $PrimFunction$ class describes
primitive functions (i.e. a function not implemented in Core', but at the machine level)
\item{Other Values}
\end{itemize}

\subsubsection{Substitution}

\subsubsection{Var}

\subsubsection{NumberedVar}

\subsubsection{Application}

\subsubsection{Thunk}

\subsection{Rule}

One rule of a user-defined function.

\subsection{StackElement}

\subsubsection{CopyStackElement}

\subsubsection{UpdateStackElement}

\section{Global functions}

\subsection{$make\_arg\_subclass$}

\subsection{$make\_constructor$}

Creates a new constructor.

\subsection{$constr$}

\subsection{$enum$}

\subsection{$function$}


\subsection{$make\_application$}


\subsection{The $evaluate\_hnf$} function simply asserts that the object to be evaluated
is an $Application$, and then calls $main\_loop$

\subsection{$main\_loop$}

\section{Primitives}




\section{Extensions}

\subsection{JSCore parser}

\subsection{Partial function application}

The following extensions was made to the Core' interpreter to aid in the conversion from
Core to Core':

\begin{itemize}
\item Added a function $make\_partial\_app$ that takes a $PartialApp$ and a argument, and returns
a $PartialApp$ if the numbers of arguments collected does not match the number of arguments required,
otherwise, it returns an $Application$
\end{itemize}


\subsection{GHC Haskell libraries}



\subsection{Primitives}






\chapter{From Core to Core'}


\section{The Case statement}


\section{Partial function application}


\section{Getting the primitive types}







\chapter{Primitives and Libraries}
\label{chap:prims}



\section{The initial plan}

GHC uses a module called GHC.Prim, located in the ghc-prim package. This
module is generated automatically from a file called "primops.txt.pp", this file
contains information about the machine representations of the various types, as
well as the operations that can be performed on them.

Initially, it was assumed that it would be trivial to translate the Haskell
libraries used by GHC into External-Core using GHC. This way, it would only 
be necessary to implement the primitive types and functions.

The External-Core functionality of GHC has bitrotted, and because of this, it can't be
used to create External-Core files for all the GHC libraries. This is a GHC bug. 
Although it produces correct
External-Core for some Haskell code, it does not for all. Since many crucial library files
can not be created External-Core for, we can not continue to use this approach.

\section{Our approach}

Currently, GHC is used to parse, typecheck, desugar and generate the 
External-Core representation of our program. By implementing the
primitive types and library functionality in Python, we are able to
get some simple programs running.


\section{Why it does not work}

% TODO: Libraries use Haskell code not supported by the External-Core libraries.

The initial reasoning for using External-Core was as follows;
it has a clearly defined syntax that is easy to work with, and it was 
assumed that it was decently supported. Unfortunately, there are bugs in
the parts of GHC responsible for generating External-Core, this in turn
results in the fact that we are unable to create External-Core files for the
necessary libraries.


\section{How to fix it}

% TODO: Should i link to the bugs ???

GHC includes a library, it should be doable to use this library to get
GHCs own internal Core representation when compiling a file. Using this 
representation, it should be easy to create a JSCore-like format to use.

However, to do this, the step performing this must be a linked into the
actual GHC binary, as it is not possible to compile the Base package without
actually compiling GHC. This is also due to a bug in GHC.

\section{Our current implementation}


% TODO: Present all library functionaliyt implemented !

Following is a description of some of the most notable implemented library 
functions.

All the builtin functionality is located in the "toplevel/pyhaskell/builtin/"
folder. When describing the implementation of a Value or a Function, we note
the Python file and the Haskell module it refers to.

\subsection{Primitive Values}

A few primitive Value types has been implemented. The ones that are currently
tested are Char and Int. See listing \ref{lst:int} for an example of a 
Value implementation. 

% TODO: _immutable_fields_ = ["value"] hit to the translator

\begin{figure}[H]
\lstset{ %
language=Python,
caption=Python class implementing the Haskell Int Value.,
label=lst:int
}
\begin{lstlisting}
class Int(haskell.Value):
    _immutable_fields_ = ["value"]

    def __init__(self, integer):
        assert isinstance(integer, int)
        self.value = integer

    def match(self, other, subst):
        value = other.getvalue()
        if value:
            assert isinstance(value, Int)
            if self.value == value.value:
                return haskell.DEFINITE_MATCH
            return haskell.NO_MATCH
        return haskell.NEEDS_HNF

    def __eq__(self, other):
        return (isinstance(other, Int) and self.value == other.value)

    def __ne__(self, other):
        return not (self == other)

    def tostr(self):
        return str(self.value)
\end{lstlisting}
\end{figure}

\subsection{Lists}

The cons (or, ":") constructor. A constructor is created with the "haskell.constr()"
function.

A Python-decorator is used to turn a Python function into a Haskell primitive function.
The decorator "@mod.expose(name, arity)" implemented in the Module class, turns a Python 
function into a haskell function by creating a PrimFunction and adding it to the 
variable-dictionary of the "mod" module. See \ref{lst:cons} for an example.

\begin{figure}[H]
\lstset{ %
language=Python,
caption=Implementation of the Haskell ++ (concatenation) operator.,
label=lst:cons
}

\begin{lstlisting}
mod = module.CoreMod("base:GHC.Base")

@mod.expose("++", 2)
def concatenation(args):
    def conc(a, b):
        if len(a.getargs() ) > 1:
            return haskell.constr(":", a.getarg(0), conc(a.getarg(1), b) )
        else:
            return b

    a, b = args
    t = conc(a,b)

    return t 
\end{lstlisting}
\end{figure}


The end of a list refors to a "[]" constructor (haskell.constr("[]")).


\subsection{unpackCString}

The unpackCString function is implemented simply. \ref{lst:unpack}




\begin{figure}[H]
\lstset{ %
language=Python,
caption=Implementation of the unpackCString function.,
label=lst:unpack
}

\begin{lstlisting}
mod = module.CoreMod("ghc-prim:GHC.CString")

@mod.expose("unpackCString#", 1)
def unpackCString(args):
    a = args[0]
    b = str(a.value)
    t = types.zmzn 
    for i in range(len(b)-1, -1, -1):
        c = haskell.constr( "Czh", Char(b[i]) )
        t = haskell.constr( ":", c, t )
    return t 
\end{lstlisting}
\end{figure}




 

\subsection{putStrLn}


\begin{figure}[H]
\lstset{ %
language=Python,
caption=Implementation of the putStrLn function.,
label=lst:put
}

\begin{lstlisting}
mod = module.CoreMod("base:System.IO")

@mod.expose("putStrLn", 1)
def putStrLn(args):
    t = args[0]
    while len(t.getargs()) > 1:
        sys.stdout.write( t.getarg(0).getarg(0).value )
        t = t.getarg(1)
    sys.stdout.write("\n")
    return args[0]
\end{lstlisting}
\end{figure}




\subsection{TopHandler}

base:GHC.TopHandler ....

runMainIO ....

wrapper for:
main ....

\subsection{The Num class}

The Num class implements some generic functions. The ones implemented
in our Python library is multiplication, subtraction and addition. These
are pretty much implemented in the same way, so we only present the
multiplication function as an example here, see listing \ref{lst:mul}.
Note that the function takes 3 arguments, the first argument 
(in the fibonacci test case, \$fNumInt) is used to tell the
generic multiplication function which specific function to use.
See section \label{subsec:classes} for a description. In our 
implementation however, the first argument is just an Int that we 
check against.

\begin{figure}[H]
\lstset{ %
language=Python,
caption=Implementation of the generic multiplication function.,
label=lst:mul
}

\begin{lstlisting}
@mod.expose("*", 3)
def multiply(args):
    ty, a, b = args
    if ty == mod.qvars["$fNumInt"]:
        return haskell.make_application(izhconstr,
            [haskell.make_application(prim.multiply, [a.getarg(0), b.getarg(0)])])
    else:
        raise NotImplementedError
\end{lstlisting}
\end{figure}






\section{Pipeline}
\label{chap:pipe}

\subsection{Haskell to extcore}

When running a Haskell file using this system, the file is first processed
by GHC.

\subsection{Extcore to JSCore}

\begin{itemize}
\item extcore
\item linkcore ?
\end{itemize}

\subsection{JSCore to Core'}

\subsection{Parsing and interpretation}




\section{Testing}
\label{chap:test}

\subsection{Current testing of interpreter}

The current testing used on the project is done by a python-script that performs
the following actions:

\begin{itemize}

\item For all folders in folder 'test':

\begin{itemize}

\item Compile haskell program with GHC and dumping to external-core

\item Run the core2js program on resulting external-core file

\item Run the main interpreter on the JSCore file.

\end{itemize}

\item Dump return-values of programs.

\end{itemize}

Our testing should be extended by checking if the output is correct.


\subsection{GHC test suit}

GHC uses a testsuit. Getting the interpreter to pass these tests are the
ultimate goal, however, the project is not there yet.


% References
\clearpage
\bibliographystyle{plain}
\bibliography{papers}

\end{document}
