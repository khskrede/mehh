
\chapter{Introduction}

This chapter discusses the motivation behind the project and 
presents a description of the project and of the work done. It 
is concluded by introducing the remaining chapters.

\section{Motivation and project description}

The project aims to investigate the feasibility of JIT (Just-in-time) 
compilation of a strongly-typed purely-functional language. Since
programs written in such a language can be heavily optimized at 
compile time, it is uncertain whether such programs can benefit from
JIT compilation. However, a JIT compiler has a lot more information to
work with than a static compiler. 

To test this, the techniques 
used by the PyPy (Python in Python) project are applied to the Haskell 
programming language. By implementing the back-end for a Haskell compiler 
in RPython (restricted Python), and using GHC (the Glasgow Haskell Compiler) 
as a front-end, a full Haskell JIT compiler can be implemented. An 
interpreter for a language very similar to the intermediate language used
by GHC was already implemented, and this is the base for this project.
Although it is already possible to have JIT compilation with GHC through
its LLVM back-end, the PyPy approach will interpret code at a much higher level.

We focus on implementing a serializer from Haskell to an intermediate
format using GHC, and a deserializer from that format to the interpreter. The 
interpreter is for a language similar to Core (the intermediate format used by GHC.)
By implementing some simple programs in Haskell, and running them through our 
compilation system, 
we hope to show that the methods of the PyPy project can be successfully applied 
to pure functional languages such as Haskell.

\section{Contributions}
The contributions of this paper is a description of Haskell-Python (an
interpreter for Core' (a lambda-calculus inspired by Haskell)), and a system for
translating Haskell programs into Core'. In addition to this, the paper 
presents a description of the full compilation system in its current state,
and a plan for the future development of the system based on the 
successes and failures so far.

Following is a listing describing how the work in this project has been
partitioned.

The following work had already been done:
\begin{itemize}

\item The Haskell-Python\cite{haskellpython} interpreter was written.

\end{itemize}

These parts were the result of an earlier stage of the project:
\begin{itemize}

\item An investigation into the use of GHC as a front-end for the compiler.
This resulted in the JSCore intermediate language, but was unsuccessful
at creating JSCore files from the GHC Haskell libraries.

\item A parser for the JSCore language was implemented; however, this 
parser was only successful for a very small subset of the JSCore language.

\item A simple system for testing functionality was implemented.

\end{itemize}

For this project, the following has been done:
\begin{itemize}

\item Another attempt was given at the creation of JSCore for the GHC 
Haskell libraries, but was still unsuccessful. The reason for this being
bugs in GHC. Specifically, bugs in the code that dumps the External-Core
files.

\item Various Haskell values and functions have been implemented at a higher
level. Due to the fact that we could not successfully create intermediate
files for the Haskell
libraries we had to implement the functionality at a higher level, i.e. 
Python.

\item Based on the gained understanding of the Core language from the
previously mentioned work, the parser has been improved. The result of 
the improvements to the parser, with the implementation of the libraries,
has resulted in the successful execution of some less trivial programs,
such as a naive recursive implementation of fibonacci.

\end{itemize}

In addition to this, some work has been done in parallell by E.W.Thomassen,
most notably:
\begin{itemize}

\item Refactoring of the code written previously, in order to make it better 
match other PyPy projects.

\item Rewriting the Python code into RPython, such that it can be translated by
the PyPy toolchain.

\item He has also changed quite a bit of the functionality for the better,
but details of this work can be found in his report.

\end{itemize}

% Write about the use-cases of laguages like Haskell, and the benefits of 
% Jit compilation

% + Research blahblah... Just to see if it works well.

% Structure of the paper. TODO
\section{Structure of the paper}
In chapter \ref{chap:back} some background information is presented, including 
some terms and concepts. 
%
A description of the 
Haskell-Python interpreter is given in chapter \ref{chap:hs}.
%
Chapter \ref{chap:rewrite} goes into detail regarding the intermediate 
languages, and the mapping between them.
%
%In chapter \ref{chap:prims} the Haskell libraries and necessary primitives
%are discussed.
%
%Chapter \ref{chap:pipe} describes the entire pipeline of the compilation system 
%at a high level.
%
%Chapter \ref{chap:test} describes the test-system used.
%
Chapter \ref{chap:impl} contains a description of the compilation system.
%
In chapter \ref{chap:similar} a brief discussion of similar work is presented.
%
Everything is wrapped up in section \ref{chap:conc}, which discusses 
the results and future work.
