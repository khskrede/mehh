

\chapter{Introduction}


\section{Project description}

The project aims to investigate the feasibility of JIT compilation of functional 
languages. To achieve this, the techniques used by the PyPy project is applied to 
Haskell.


\subsection{Python in Python}

The \emph{PyPy} (Python in Python) project is successfull at boosting the speed 
of the Python programming language
through the use of a Just-In-Time compiler. The PyPy project contains a tracing JIT for 
RPython (a restricted proper subset of Python.) The RPython program is translated 
into a C program containing a JIT compiler. 

A tracing JIT compiler works by tracing the execution of the program, and then feeding
this information back in order to exploit it for increased performance. The PyPy 
project manages to trace the meta-level; by tracing the interpreter written in RPython,
the program being interpreted are also, in effect being traced.



\subsection{Haskell, GHC and External-Core}

GHC (The Glasgow Haskell Compiler) uses an intermediate language throughout it's 
simplification phase. The extcore project presents a formal definition of the syntax 
of this language. And in addition to this, enables the representation to be exported 
to files. The idea is that this allows compiler implementors and researchers to use GHC
as a front-end for Haskell compilers. Before outputting external-core files,
the Haskell files are typechecked, desugared and simplified. \cite{tolmach2010ghc}

\begin{comment}
The linkcore project implements a linker for core programs, i.e. it transforms
a single Haskell module into a single closed external-core module. In addition to
this, since the linker requires external-core representation of the ghc-libraries,
it also contains instructions on how to create these. 
\end{comment}




\section{Similar work}


\subsection{External-core and Linkcore}



\subsection{Other PyPy interpreters}

\subsubsection{PyPy Prolog}


\section{Motivation}

% Write about the use-cases of laguages like Haskell, and the benefits of 
% Jit compilation

% + Research blahblah... Just to see if it works well.


