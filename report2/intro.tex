
\chapter{Introduction}

\section{Motivation and project description}

% Why are we doing what we are doing?
The project aims to investigate the feasibility of JIT (Just-in-time) 
compilation of a strongly-typed purely-functional language. Since
programs written in such a language can be heavily optimized at 
compile time, it us uncertain whether such programs can benefit from
JIT compilation. However, a JIT compiler has a lot more information to
work with than a static compiler. 

To test this, the techniques 
used by the PyPy (Python in Python) project is applied to the Haskell 
programming language. By implementing the back-end for a Haskell compiler 
in RPython (restricted Python), and using GHC (the Glasgow Haskell Compiler) 
as a front-end, we can effectively implement a full Haskell JIT compiler. An 
interpreter for a language very similar to the intermediate language used
by GHC was allready implemented, and this is the base for our work.

% What are we doing?
We focus on implementing a serializer from Haskell to an intermediate
format using GHC, and a deserializer from that format to the interpreter. The 
interpreter is for a language similar to Core (an intermediate format used by GHC.)
By implementing some simple programs in Haskell, and running them in this way, 
we hope to show that the methods of the PyPy project can be applied to pure 
functional languages such as Haskell.

% What are the contributions of this paper?
The contributions of this paper is a sytem for translating Haskell programs
into Core', and a description of the full compilation system at its current
state, including the Core' interpreter (Haskell-Python). In addition to this,
we present a plan for the future development of the system.


% Structure of the paper. TODO


% Write about the use-cases of laguages like Haskell, and the benefits of 
% Jit compilation

% + Research blahblah... Just to see if it works well.


