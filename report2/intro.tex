
\chapter{Introduction}

\section{Motivation and project description}

% Why are we doing what we are doing?
The project aims to investigate the feasibility of JIT (Just-in-time) 
compilation of a strongly-typed purely-functional language. Since
programs written in such a language can be heavily optimized at 
compile time, it us uncertain whether such programs can benefit from
JIT compilation. However, a JIT compiler has a lot more information to
work with than a static compiler. 

To test this, the techniques 
used by the PyPy (Python in Python) project is applied to the Haskell 
programming language. By implementing the back-end for a Haskell compiler 
in RPython (restricted Python), and using GHC (the Glasgow Haskell Compiler) 
as a front-end, we can effectively implement a full Haskell JIT compiler. An 
interpreter for a language very similar to the intermediate language used
by GHC was allready implemented, and this is the base for our work.
Although it is already possible to have JIT compilation with GHC through
its LLVM backend, the PyPy approach will interpret code at a much higher level.

% What are we doing?
We focus on implementing a serializer from Haskell to an intermediate
format using GHC, and a deserializer from that format to the interpreter. The 
interpreter is for a language similar to Core (an intermediate format used by GHC.)
By implementing some simple programs in Haskell, and running them through our 
compilation system, 
we hope to show that the methods of the PyPy project can be successfully applied 
to pure functional languages such as Haskell.

% What are the contributions of this paper?

\section{Contributions}
The contributions of this paper is a description of Haskell-Python (an
interpreter for Core' (a lambda-calculus inspired by Haskell)), and a system for
translating Haskell programs into Core'. In addition to this, the paper 
presents a description of the full compilation system in its current state,
and a plan for the future development of the system based on the 
successes and failures so far.

% Write about the use-cases of laguages like Haskell, and the benefits of 
% Jit compilation

% + Research blahblah... Just to see if it works well.

% Structure of the paper. TODO
\section{Structure of the paper}
In section \ref{chap:back} we present background information, including 
some terms and concepts and similar work. 
%
A description of the 
Haskell-Python interpreter is given in section \ref{chap:hs}.
%
Chapter \ref{chap:rewrite} goes into detail regarding the intermediate 
languages, and the mapping between them.
%
In section \ref{chap:prims} the Haskell libraries and necessary primitives
are discussed.
%
Chapter \ref{chap:pipe} describes the entire pipeline of the compilation system 
at a high level.
%
Chapter \ref{chap:test} describes the test-system used.
%
Everything is wrapped up in section \ref{chap:conc}, which discusses the 
results and future work.
