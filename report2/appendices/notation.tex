
\chapter{Notation}

\section{Logical deductions (inference rules)}

Inference rules are rules one may follow to make sure that the conclusions one get are


\subsection{Example: Modus ponens}

Following is the modus ponens inference rule as an example of the 
notation used in this paper.

\begin{equation}
\dfrac{P, \;\;\; P \;\; implies \;\; Q}{Q}
\end{equation}

This can be read as; a proof of P together with a proof that P implies Q, is a proof of Q.

When the statements on the top of the line (\emph{antecedents}) are proved, then we can consider
the statements below the line (\emph{conclusion} or \emph{consequent}) to also be proved. 

In this paper, it would make most sence to use the word \emph{consequent}, as the notation is used to
describe reduction rules in the \emph{lambda calculus}. 

\subsection{Example: Function application}

The \emph{modus ponens} inference rule actually
corresponds to the \emph{application rule} of the \emph{lambda calculus}. 
Thus, we can read the example as:

\begin{center}
"Given P, and a function mapping P to Q, infer Q." 
\end{center}

This is written as:

\begin{equation}
\dfrac{\Gamma : e \Downarrow \Delta : \lambda y . e' \;\;\;\;\;\;\;\;\; \Delta : e' [x/y] \Downarrow \Theta : z }{ \Gamma : e \; x \Downarrow \Theta : z}
\end{equation}

And we can relate it to the modus ponens as following:

P is the existance of a reduction ($\Downarrow$) from the expression $e$ in the context $\Gamma$ to the 
lambda abstraction $\lambda y . e'$ in the context of $\Delta$.

The implication from P to Q can be read as the existance of the application of the value $x$ to the 
variable $y$ in the expression $e'$ in the context of $\Delta$ is reduced to the value $z$ in the context of
$\Theta$.

And the \emph{consequent}, Q, is thus;

in the paper. 

In this specific description: 
\begin{center}

$P \cong \Gamma : e \Downarrow \Delta : \lambda y . e'$ 
\\
and 
\\
$P \; \; implies \; \; Q \cong \Delta : e' [x/y] \Downarrow \Theta : z$ 

\end{center}



\section{Formal grammars}


