\chapter{Similar work}
\label{chap:similar}

In addition to the Python implementation, PyPy implements a low-level 
hardware emulator (PyGirl), a PHP interpreter, and a Prolog interpreter. 
Various other experiments have also been created by the PyPy team. This
chapter is a brief discussion of these projects and experiments. 

Some work has also been done on the GHC side, among others, a LLVM
(Low Level Virtual Machine) back-end has been implemented.

% TODO !
\section{The GHC LLVM back-end}

The LLVM (Low Level Virtual Machine) is a framework for the optimization of 
programs from the compilation phase to runtime. The LLVM provides high-level information 
to the compilation system during compile-time, run-time and in idle time between
runs. By creating code generators for the virtual instruction set supported by
LLVM, implementors can take full advantage of its features.
\cite{lattner2004llvm}

GHC can generate LLVM code from Cmm (C minus minus; is a low-level imperative
language with an explicit stack). In some cases the 
LLVM back-end can produce significantly faster code than the traditional route. 
\cite{marlow2012glasgow, terei2010llvm}

\section{PyPy Prolog}

In addition to the implementation of Python, PyPy has also shown that its techniques
are applicable to other languages. The Prolog VM is an example of this. Implementations
of Prolog are usually written in low-level languages such as C, this usually results in
good performance, but means they are difficult to write and maintain. The PyPy Prolog 
interpreter clearly outperforms other Prolog interpreters written in other high-level
languages, and it also outperforms state-of-the-art Prolog VMs at specific benchmarks,
which shows that other Prolog implementations can benefit from the techniques used by
PyPy. \cite{bolz2010towards}

% TODO !
\section{HappyJIT}

PHP (Hypertext Preprocessor) is a language used to develop the server-side of 
websites. The users request for a website is received by the server, the PHP script
then executes the request, often involving querying a database and then generating 
the actual HTML for the user. Increasing the effectiveness of this process would
reduce the time it takes for a user to have a website request answered. 
The HappyJIT project implements a PHP interpreter in RPython, this interpreter is 
translated by the PyPy translator into a tracing-JIT. The approach show that 
the techniques significantly improve the performance of several common use cases.
\cite{homescu2011happyjit}

% TODO 
\section{PyGirl}

As a case study, PyGirl implements an emulator for the Nintendo Game Boy. The project 
shows the feasibility of implementing a low-level VM for hardware in a high-level 
language to improve portability, and reduce complexity. The project shows that the
reduction in implementation complexity with this approach is substantial, 
and that the performance loss can be insignificant.
\cite{bruni2009pygirl}


