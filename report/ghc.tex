\section{GHC}

\emph{Haskell} is a \emph{strongly-typed non-strict purely-functional} programming language, it will not be
described in detail here, since Haskell is not the language we focus on. See \cite{hudak1992report}
for an introduction to Haskell. 

Core is an intermediate language used by the Glasgow Haskell Compiler\cite{ghc},
and it is this language we wish to interpret. Core is a desugared version of Haskell, things like pattern matching
and list comprehensions are transformed out to simpler constructs.\cite{jones1994compilation}

\subsection{Compilation by transformation}

GHC uses a compilation idiom called \emph{compilation by transformation}. The idea is to repeatedly perform 
correctness-preserving transformations to the program. Ideally, these transformations will result
in a semantically equal program that executes more quickly or in less space. Such transformations seem
to fall into two categories:

\begin{itemize} 
\item{\emph{Glamorous transformations}} are global, sophisticated, intellectually satisfying transformations,
sometimes guided by some interesting kind of analyzis.
\item{\emph{Humble transformations}} are small, simple, local transformations. Individually they look very trivial.
\end{itemize}

In the Glasgow Haskell Compiler, all humble transformations are done by the \emph{simplifier}. 
\cite{jones1994compilation} The simplifier repeatedly performs transformations on the Core language.


\subsection{The GHC API and usage}

...


