\section{JSON representation of Core}

\subsection{JSON}

JavaScript Object Notation (JSON) is a lightweight data interchange format.

Since a library for manipulating JSON is available for Haskell, this
makes it a good choice for the project. In addition, it is easy to parse and
the Haskell library contains a pretty-printer, making the result easier to
inspect.

For a formal definition of JSON see appendix \ref{jsongrammar}

\subsection{JSCore}

In order to work with JSON and External-core, a format was defined that
expresses the Core program in JSON notation. Most of the right hand side of 
the grammar evaluates to JSON Values. 
Even though the grammar is changed to support JSON, an effort was made to
keep it similar to the original Core grammar for easy referencing. The size
of the resulting files was not considered to be an issue. Other than the 
syntax, the JSCore language is the same as External-core.

For a formal definition of JSCore see appendix \ref{jscoregrammar}

