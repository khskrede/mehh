
\section{External-core}

% New

The Core language is an intermediate language used by GHC. It is the internal
program representation in the compilers simplification phase. It appears to be a
subset of Haskell, but with explicit type annotations in the style of the polymorphic
lambda calculus ($F_w$). External-core
is an external representation of Core generated from Haskell, by using GHC with a compiler 
flag (\emph{-fext-core}). By using External-core, one
may implement just parts of a Haskell compiler, using the remaining parts from
GHC. For this project, GHC is used to generate External-core. This way, desugaring,
type checking, pattern matching and overloading is performed. The remaining task
is then to interpret the External-core representation. \cite{tolmach2010ghc}

The GHC-generated External-core uses z-encoding for special characters in names 
(variables, constructors, ...), for
a definition of the z-encoding, see appendix \ref{zencoding}. This is a good thing, as much
of the functionality represented by these names has to be implemented in RPython,
and special characters (like \#{}, or +) is not allowed in Python names (function
identifiers, class identifiers, ...).

Without using GHC to produce External-core, linking code into GHC would be an 
optional way of achieving this, which would be a difficult and large task.
Or, the GHC API could be used to do the same task more cleanly. \cite{tolmach2010ghc}

The initial starting point of this project was to use the GHC API, the reasoning
was that External-core is not fully parenthesized and more tricky to parse. Thus,
generating a fully parenthesized and more machine-readable format would make sense.
However, it turned out that this too was a complicated task. As the internal
datatypes of GHC representing Core does not match the description of External-core. 
The choice was
then made to use GHC-generated External-core as the base for creating a new intermediate
format that could be easily generated using available packages for manipulating
External-core (\emph{extcore} is used for this).

For a formal definition of External-core, see appendix \ref{coregrammar}.

\subsection{Informal semantics}

%\begin{itemize}

\subsubsection{Module}

The first construct in the External-core representation of a program, is the \emph{module}. 
This construct is represented by a \emph{module identifier}, followed by a list of 
\emph{type definitions} and \emph{value definitions} respectively.

The \emph{module} directly corresponds to Haskell source modules. The \emph{module identifier}
contains information of what package it belongs to, followed by its name. \emph{Identifiers} 
that are defined at the top level of the module can be internal or external. External identifiers
can be referenced from other modules in the program, internal identifiers can not.\cite{tolmach2010ghc}


\subsubsection{Type definition}

A \emph{type definition} can be an \emph{algebraic-type}, or a \emph{newtype} construct. 

An \emph{algebraic-type} is represented by the "\%data" keyword, a
\emph{type constructor} and a \emph{type binder}, 
followed by a list of \emph{constructor definitions}.
Each new \emph{algebraic type} introduces a new \emph{type constructor} 
and a set of one or more \emph{constructor definitions}. 

A \emph{newtype} construct is represented by the "\%newtype" keyword, two
\emph{type constructors}, a set of \emph{type binders} and a \emph{type}.\cite{tolmach2010ghc}


\subsubsection{Constructor definition}

A \emph{constructor definition} is represented by a \emph{data constructor}, followed by a 
\emph{type binder} and one or more \emph{atomic types}.





\subsubsection{Value definition}

A \emph{value definition} can be \emph{recursive} or \emph{non-recursive}. A 
\emph{recursive value definition} is represented by the "\%rec" keyword and one or more 
\emph{non-recursive value definitions}. A \emph{non-recursive value definition} 
is represented by a \emph{variable identifier} followed by a \emph{type} and 
an \emph{expression}.\cite{tolmach2010ghc}



\subsubsection{Atomic expression}

An \emph{atomic expression} can be either of the following; a \emph{variable}, 
\emph{data constructor}, \emph{literal} or a \emph{nested expression}.\cite{tolmach2010ghc}




\subsubsection{Expression}

An \emph{expression} can be either of the followin; an \emph{atomic expression},
\emph{application}, (lambda) \emph{abstraction}, \emph{local definition}, 
\emph{case expression},
\emph{type coercion}, \emph{expression note}, \emph{external reference}, 
\emph{dynamic external reference}, or an \emph{external label}. See appendix 
\ref{coregrammar} table \ref{core} for their definitions.\cite{tolmach2010ghc}



\subsubsection{Argument}

An \emph{argument} can be either a \emph{type argument} or a \emph{value argument}. 
A \emph{type argument} is simply an \emph{atomic type}, and a \emph{value argument} is an 
\emph{atomic expression}.

Arguments are applied to \emph{atomic expressions} to make \emph{applications}.\cite{tolmach2010ghc}


\subsubsection{Case alternative}

A \emph{case alternative} can be either of the following; a 
\emph{constructor alternative}, a \emph{literal alternative} or a 
\emph{default alternative}.\cite{tolmach2010ghc}



\subsubsection{Binder}

A \emph{binder} can be either a \emph{type binder} or a \emph{value binder}. It binds
a \emph{value} to a \emph{type}, or a \emph{type} to a \emph{kind}.
A \emph{type binder} is represented by a "@" and a \emph{type binder}.
A \emph{value binder} is simply represented as a \emph{value binder}.




%\subsubsection{Type binder}

A \emph{type binder} can either be a \emph{type variable} (implicitly of kind *), or a 
\emph{type variable} followed by a \emph{kind} (explicitly kinded).







%\subsubsection{Value binder}

A \emph{value binder} is a \emph{variable} followed by a \emph{type}.\cite{tolmach2010ghc}





\subsubsection{Literal}

A \emph{literal} can be either an \emph{integer}, \emph{rational}, \emph{character} 
or \emph{string}. See appendix \ref{coregrammar} table \ref{core} for their definitions.\cite{tolmach2010ghc}






\subsubsection{Atomic type}

An \emph{atomic type} can be either a \emph{type variable}, \emph{type constructor} 
or a \emph{nested type}.\cite{tolmach2010ghc}





\subsubsection{Basic type}

A \emph{basic type} can be either of the following; an \emph{atomic type}, 
\emph{type application}, \emph{transitive coercion}, \emph{symmetric coercion},
\emph{unsafe coercion}, \emph{left coercion}, \emph{right coercion}, or an
\emph{instantiation coercion}.\cite{tolmach2010ghc}




\subsubsection{Type}

A \emph{type} can be either a \emph{basic type}, a \emph{type abstraction} or a
\emph{arrow type construction}.

Types are built from \emph{type constructors} and \emph{type variables} using
\emph{type application} and \emph{universal quantification}. There are a number
of primitive \emph{type constructors} defined in the "GHC.Prim" module. 
\emph{algebraic-type} and \emph{newtype} constructs introduce new 
\emph{type constructors}. The \emph{type constructors} are distinguished by name
only.\cite{tolmach2010ghc}





\subsubsection{Atomic kind}

An \emph{atomic kind} can be either of the following; a \emph{lifted kind} (*), 
\emph{unlifted kind} (\#{}), \emph{open kind} (?), \emph{equality kind}, or a
\emph{nested kind}.\cite{tolmach2010ghc}





\subsubsection{Kind}

A \emph{kind} can be either an \emph{atomic kind} or an \emph{arrow kind}.
An \emph{arrow kind} is an \emph{atomic kind} followed by an arrow (-$>$) and
another \emph{kind}.\cite{tolmach2010ghc}





\subsection{Evaluation of a program}

A program is evaluated by reducing the expression "main:ZCMain.main" (note that
qualified names in a module named $m$ must have module-name $m$, this is the only exception) 
to \emph{weak-head-normal-form} (WHNF), i.e. a primitive value, lambda abstraction, or 
fully applied data constructor. A heap is used to make
sure evaluation is shared. The heap contains two types; a \emph{thunk}, 
or a \emph{WHNF}. A thunk is an unevaluated
expression, also called a \emph{suspension}. A \emph{WHNF} is an evaluated expression, 
the result of evaluating a \emph{thunk} is a \emph{WHNF}. \cite{tolmach2010ghc}

