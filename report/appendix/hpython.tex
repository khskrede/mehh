
\section{Haskell-Python}

\subsection{Implemented classes}

\begin{itemize}

\item Symbol: 			 A cached symbol that can be compared by identity (which is not true for strings). 
\item HaskellObject: 		 Base class for all objects that the interpreter handles. 
\item Value:			 Base class for evaluated values (i.e. already in head-normal form). 
\item Constructor:		 A constructor. This is an abstract base class, there are subclasses generated below for various numbers of arguments.
\item ConstructorN:		 
\item AbstractFunction:	 
\item Function:		 	 A user-defined function, i.e. written in Haskell 
\item Rule:			 One rule of a user-defined function. 
\item Substitution:		 The body of a function with numbered variables substituted by values. 
\item PrimFunction:		 A primitive function, i.e. one not implemented in Haskell but at the machine level. 
\item Var:			 A variable.
\item NumberedVar:		 
\item Application:		 A function application. This is an abstract base class, there are subclasses generated below for various numbers of arguments. 
\item ApplicationN:		 
\item Thunk:			 An unevaluated function application. 
\item StackElement:		 Base class of the stack elements of the evaluation stack. 
\item CopyStackElement:	 	Need to copy the top of the stack. 
\item UpdateStackElement:	 Need to update the thunk stored in this after its content has been evaluated. 

\end{itemize}

\subsection{Implemented functions}

The interpreter implements the following functions:

\begin{itemize}

\item make\_arg\_subclass(n, base)
\item make\_constructor(function, args)
\item make\_constr(name, *args)
\item enum(rule, subst)
\item function(name, rules, recursive=False): Takes a function name, and a set of paramters (rules), to generate the Rules object necessary to create a Function object. Returns a Function object.
\item make\_application(function, args)
\item get\_printable\_location(function)
\item evaluate\_hnf(obj): Wrapper for main\_loop, with assertion. Used for testing.
\item main\_loop(expr): Takes the main \emph{function application} as argument, and evaluates
the program

\end{itemize}


