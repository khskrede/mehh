\section{PyPy}

\subsection{A quick overview}

The PyPy project is basically two things:

\begin{enumerate}

\item The RPython toolchain; a set of compiler tools for programs written in 
RPython.

\item An implementation of Python using these tools.

\end{enumerate}

In this paper PyPy refers to former.

The basic concept of PyPy is to use a high-level language to allow for rapid
development of interpreters for a variety of platforms. By implementing a compiler
for RPython, interpreters for other languages can be written in RPython and 
compiled to any platform supported by the PyPy toolchain.

\subsection{RPython}

RPython is a restricted proper subset of Python, this enables easy analysis 
as well as efficient compilation.

\subsection{Translation toolchain}


\subsection{Just-in-time compilation}

The reason why PyPy is able to compete with other language implementations
on speed is it's tracing, or rather meta-tracing, JIT. The JIT is implemented
for RPython, but through a set of compiler hints, it is able to trace the 
execution of the application interpreted by the RPython program.

An introduction to virtual machine (VM) construction with python \cite{pypy}.

