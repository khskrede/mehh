\section{Haskell-Python}

The Haskell-Python interpreter contains the following classes:

\begin{longtable}{|p{1.5in}|p{3in}|}
\hline
Class name 	& Docstring \\
\hline \hline
Symbol 			& A cached symbol that can be compared by identity (which is not true for strings). \\
\hline
HaskellObject 		& Base class for all objects that the interpreter handles. \\
\hline
Value			& Base class for evaluated values (i.e. already in head-normal form). \\
\hline
Constructor		& A constructor. This is an abstract base class, there are subclasses generated below for various numbers of arguments.\\
\hline
ConstructorN		& \\
\hline
Integer			& \\
\hline
AbstractFunction	& \\
\hline
Function		& A user-defined function, i.e. written in Haskell \\
\hline
Rule			& One rule of a user-defined function. \\
\hline
Substitution		& The body of a function with numbered variables substituted by values. \\
\hline
PrimFunction		& A primitive function, i.e. one not implemented in Haskell but at the machine level. \\
\hline
Var			& \\
\hline
NumberedVar		& \\
\hline
Application		& A function application. This is an abstract base class, there are subclasses generated below for various numbers of arguments. \\
\hline
ApplicationN		& \\
\hline
Thunk			& An unevaluated function application. \\
\hline
StackElement		& Base class of the stack elements of the evaluation stack. \\
\hline
CopyStackElement	& Need to copy the top of the stack. \\
\hline
UpdateStackElement	& Need to update the thunk stored in this after its content has been evaluated. \\
\hline

\end{longtable}

\subsection{Our extensions}

The extensions to Haskell-Python implemented in this project, is the deserializer.
