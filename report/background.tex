
\section{Background}

\subsection{PyPy}

\subsubsection{A quick overview}

The PyPy project is basically two things:

\begin{enumerate}
\item the RPython toolchain, written in Python, is a set of compiler tools for 
programs written in RPython.
\item and an implementation of Python using these tools.
\end{enumerate}

In this paper PyPy refers to former.

The basic concept of PyPy is to use a high-level language to allow for rapid
development of interpreters for a variety of platforms. By implementing a compiler
for RPython, interpreters for other languages can be written in RPython and 
compiled to any platform supported by the PyPy toolchain. Supported platforms include
CLI and JVM. \cite{ancona2007rpython}

PyPy uses the meta-programming argument; if a VM (virtual machine) can be written
at a level of abstraction high enough, then it should be possible to automatically translate 
this VM to other lower-level platforms. This is what PyPy does. \cite{pypy}

\subsubsection{RPython}

RPython (Restricted Python) is a restricted proper subset of Python that it 
is able possible to perform type inference on. This means that it can be 
translated to efficient C code, and it enables easy analyzis 
as well as efficient compilation. This also means that RPython code can be
run and debugged by Python interpreters, like CPython. \cite{ancona2007rpython}

\subsubsection{Just-in-time compilation}

The JIT compiler is the reason why PyPy is able to compete with other language implementations
on speed. Or rather, it's meta-tracing JIT. The JIT is implemented
for the RPython compiler, but through a set of compiler hints, it is able to trace the 
execution of the application interpreted by the RPython program.

\subsubsection{Haskell-Python}

Haskell-Python is an interpreter for a subset of the Haskell language, called Core'.
We call it Core' here because it does not directly correspond to the Core language
used by GHC (the glasgow haskell compiler). Haskell-Python is written in RPython,
and is compiled into a JIT compiler using the RPython toolchain. Our goal for this
project is to extend Haskell-Python to use GHC as a frontend for compilation 
of Haskell programs.

\subsection{GHC}

\emph{Haskell} is a \emph{strongly-typed non-strict purely-functional} 
programming language, it will not be described in any detail here, since 
Haskell is not the language we focus on. See \cite{hudak1992report}
for an introduction to Haskell. 

Core is an intermediate language used by the Glasgow Haskell Compiler\cite{ghc},
and it is this language we wish to interpret. Core is a desugared version of Haskell, 
things like pattern matching
and list comprehensions are transformed out to simpler constructs.\cite{jones1994compilation}

\subsection{Extcore}

Extcore is a Haskell package for working with GHC's Core language. Among other things,
it implements a parser for external-core, this is the part used from extcore in this project.

\begin{comment}
\subsubsection{Compilation by transformation}

GHC uses a compilation idiom called \emph{compilation by transformation}. The idea is to repeatedly perform 
correctness-preserving transformations to the program. Ideally, these transformations will result
in a semantically equal program that executes more quickly or in less space. Such transformations seem
to fall into two categories:

\begin{itemize} 
\item{\emph{Glamorous transformations}} are global, sophisticated, intellectually satisfying transformations,
sometimes guided by some interesting kind of analyzis.
\item{\emph{Humble transformations}} are small, simple, local transformations. Individually they look very trivial.
\end{itemize}

In the Glasgow Haskell Compiler, all humble transformations are done by the \emph{simplifier}. 
\cite{jones1994compilation} The simplifier repeatedly performs transformations on the Core language.


\subsection{Extcore}

Extcore is a package for working with the external-core format.

\end{comment}
