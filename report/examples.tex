
\section{Example}


\subsection{Example: hello world}

This example program is a simple "hello world" program, as this was practically
the only program that was able to pass through the entire pipeline. Following
is the "hello world" program written in Haskell:

\lstinputlisting[language=Haskell]{"../interpreter/tests/helloworld/helloworld.hs"}

\subsubsection{Converted to Core}

After the program has passed through GHC and the External-core file
has been generated, the program looks like this:

\lstinputlisting{"../interpreter/tests/helloworld/helloworld.hcr"}

\subsubsection{Converted to JSCore}

In the next step it is parsed and dumped to JSCore:

\lstinputlisting{"../interpreter/tests/helloworld/helloworld.hcj"}

\subsubsection{JSCore graph}

Using the parsing libraries of PyPy we can generate a nice graph from the result,
directly corresponding to the resulting datastructure. 
See appendix \ref{graphs}, figure \ref{helloworldgraph}.
By simply traversing this datastructure we can generate the AST for the Core 
interpreter (Haskell-Python).

\subsubsection{Result}

The program results as expected, outputting the string "Hello, world!".









\begin{comment}

\subsection{Example 2: naive fibonacci}

The following program is a simple fibonacci program.

\lstinputlisting[language=Haskell]{"../interpreter/tests/fibonacci/fibonacci.hs"}

\subsubsection{Converted to Core}

As we can see, the simple hello world program becomes more complex when translated
to Core by GHC.

\lstinputlisting{"../interpreter/tests/fibonacci/fibonacci.hcr"}

\subsubsection{Converted to JSCore}

And translated to JSCore by our serializer:

\lstinputlisting{"../interpreter/tests/fibonacci/fibonacci.hcj"}

\subsubsection{Result}

\end{comment}
