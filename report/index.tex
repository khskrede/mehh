
\documentclass{article}

\usepackage{verbatim}
\usepackage{cite}
\usepackage{parskip}


\begin{document}

\title{Title}
\author{Author}
\maketitle

\begin{abstract}
Abstract
\end{abstract}

\clearpage


\section{Introduction}

\subsection{Haskell}

For a technical overview of the glasgow haskell compiler (ghc) \cite{ghc}.

\subsubsection{Core haskell}

\subsection{Pypy}

An introduction to virtual machine (VM) construction with python\cite{pypy}.

\subsection{Motivation}

The motiviation behind the project is to see if a statically typed functional 
programming language, haskell, can benefit from just-in-time (JIT) compilation.

\section{Implementation}

The system will use the GHC haskell library to generate the haskell Core language
and a simple haskell program to perform preprocessing on this language.

The resulting code will then be parsed by a python program that generates code for the
PyPy haskell interpreter.

\subsection{Generating Core haskell}

Core is easily generated by using the GHC Haskell api.

\subsection{Preprocessing Core}


\subsection{PyPy Interpreting preprocessed core haskell}


\section{Testing}


\section{Benchmarks}


\bibliographystyle{unsrt}

\bibliography{papers}

\end{document}
