
\documentclass{article}

\usepackage{verbatim}
\usepackage{cite}
\usepackage{parskip}
\usepackage{longtable}
\usepackage{listings}
\usepackage{graphicx}
\usepackage{float}
\usepackage{rotating}
\usepackage{fancyhdr}
%\usepackage[margin=.7in]{geometry}
\usepackage{appendix}

\usepackage{color}
\definecolor{Grey}{rgb}{0.5,0.5,0.5} 

% Make margin of captions smaller
\usepackage[margin=1cm]{caption}

% -------------------------
% Test stuff with listings
% -------------------------
\usepackage{xcolor}
\usepackage{caption}
\DeclareCaptionFormat{listing}{%
  \parbox{\textwidth}{\colorbox{Gray}{\parbox{\textwidth}{#1#2#3}}\vskip-4pt}}

\lstset{ %
basicstyle=\scriptsize,         % the size of the fonts that are used for the code
%numbers=left,                   % where to put the line-numbers
numberstyle=\scriptsize,        % the size of the fonts that are used for the line-numbers
%frame=trBL%single,                   % adds a frame around the code
frame=single,
framerule=1pt,
frameround=tttt,
%backgroundcolor=\color{red},
showstringspaces=false,
tabsize=2,                      % sets default tabsize to 2 spaces
captionpos=b,                   % sets the caption-position to bottom
breaklines=true,                % sets automatic line breaking
breakatwhitespace=false,        % sets if automatic breaks should only happen at whitespace
%title=\lstname,			% include title (file name)
xleftmargin=5pt,
xrightmargin=5pt,
%rulecolor=\color{Grey}
}

% -------------------------
% Test with section numbers in the margin
% -------------------------
%\makeatletter
%\def\@seccntformat#1{\color{Grey}\llap{\csname the#1\endcsname\quad}}
%\makeatother


% Different font in captions
\newcommand{\captionfonts}{\small}

\makeatletter  % Allow the use of @ in command names
\long\def\@makecaption#1#2{%
  \vskip\abovecaptionskip
  \sbox\@tempboxa{{\captionfonts #1: #2}}%
  \ifdim \wd\@tempboxa >\hsize
    {\captionfonts #1: #2\par}
  \else
    \hbox to\hsize{\hfil\box\@tempboxa\hfil}%
  \fi
  \vskip\belowcaptionskip}
\makeatother   % Cancel the effect of \makeatletter

% Options for footnotes
\setlength{\footnotesep}{0.5cm}
%\setlength{\skip\footins}{2cm}

\begin{document}

\pagestyle{fancy}


\title{Just-In-Time compilation of\\Haskell using PyPy and GHC}
\author{Knut Halvor Skrede}

% Frontpage
\begin{titlepage}
\maketitle

\begin{abstract}
The paper describes a system for using GHC as frontend and PyPy as backend for a 
Haskell JIT compiler. An intermediate language in JSON format based on GHC's Core
language is described. The implemented parts are a serializer written in Haskell and
a deserializer written in Python, in addition to some Haskell library functions.
The project is meant to serve as a base for further development. 
\end{abstract}

\end{titlepage}


\tableofcontents

\clearpage

\listoffigures
\listoftables

\clearpage

\setlength\LTleft{0pt}
\setlength\LTright{0pt}

% Introduction


\section{Introduction}

\subsection{Motivation and project description}

% Why are we doing what we are doing?
The project aims to investigate the feasibility of JIT (Just-in-time) 
compilation of a strongly-typed purely-functional language. Since
programs written in such a language can be heavily optimized at 
compile time, it us uncertain whether such programs can benefit from
JIT compilation. However, a JIT compiler has a lot more information to
work with than a static compiler. 

To test this, the techniques 
used by the PyPy (Python in Python) project is applied to the Haskell 
programming language. By implementing the back-end for a Haskell compiler 
in RPython (restricted Python), and using GHC (the Glasgow Haskell Compiler) 
as a front-end, we can effectively implement a full Haskell JIT compiler. An 
interpreter for a language very similar to the intermediate language used
by GHC was allready implemented, and this is the base for our work.
Although it is already possible to have JIT compilation with GHC through
its LLVM backend, the PyPy approach will interpret code at a much higher level.

% What are we doing?
We focus on implementing a serializer from Haskell to an intermediate
format using GHC, and a deserializer from that format to the interpreter. The 
interpreter is for a language similar to Core (an intermediate format used by GHC.)
By implementing some simple programs in Haskell, and running them through our 
compilation system, 
we hope to show that the methods of the PyPy project can be successfully applied 
to pure functional languages such as Haskell.

% What are the contributions of this paper?

\subsection{Contributions}
The contributions of this paper is a description of Haskell-Python (an
interpreter for Core' (a lambda-calculus inspired by Haskell)), and a system for
translating Haskell programs into Core'. In addition to this, the paper 
presents a description of the full compilation system in its current state,
and a plan for the future development of the system based on the 
successes and failures so far.

% Write about the use-cases of laguages like Haskell, and the benefits of 
% Jit compilation

% + Research blahblah... Just to see if it works well.

% Structure of the paper. TODO
\subsection{Structure of the paper}
In subsection \ref{chap:back} we present background information, including 
some terms and concepts and similar work. 
%
A description of the 
Haskell-Python interpreter is given in subsection \ref{chap:hs}.
%
section \ref{chap:rewrite} goes into detail regarding the intermediate 
languages, and the mapping between them.
%
In subsection \ref{chap:prims} the Haskell libraries and necessary primitives
are discussed.
%
section \ref{chap:pipe} describes the entire pipeline of the compilation system 
at a high level.
%
section \ref{chap:test} describes the test-system used.
%
Everything is wrapped up in subsection \ref{chap:conc}, which discusses the 
results and future work.


% Background

%\clearpage

\section{Background}

\subsection{PyPy}

\subsubsection{A quick overview}

The PyPy project is basically two things:

\begin{enumerate}
\item the RPython toolchain, written in Python, is a set of compiler tools for 
programs written in RPython.
\item and an implementation of Python using these tools.
\end{enumerate}

In this paper PyPy refers to former.

The basic concept of PyPy is to use a high-level language to allow for rapid
development of interpreters for a variety of platforms. By implementing a compiler
for RPython, interpreters for other languages can be written in RPython and 
compiled to any platform supported by the PyPy toolchain. Supported platforms include
CLI and JVM. \cite{ancona2007rpython}

PyPy uses the meta-programming argument; if a VM (virtual machine) can be written
at a level of abstraction high enough, then it should be possible to automatically translate 
this VM to other lower-level platforms. This is what PyPy does. \cite{pypy}

\subsubsection{RPython}

RPython (Restricted Python) is a restricted proper subset of Python that it 
is possible to perform type inference on. This means that it can be 
translated to efficient C code, and it enables easy analyzis 
as well as efficient compilation. This also means that RPython code can be
run and debugged by Python interpreters, like CPython. \cite{ancona2007rpython}

\subsubsection{Just-in-time compilation}

The JIT compiler is the reason why PyPy is able to compete with other language implementations
on speed. Or rather, it's meta-tracing JIT. The JIT is implemented
for the RPython compiler, but through a set of compiler hints, it is able to trace the 
execution of the application interpreted by the RPython program.

\subsubsection{Haskell-Python}

Haskell-Python is an interpreter for a subset of the Haskell language, called Core'.
We call it Core' (Core marked) here because it does not directly correspond to 
the Core language
used by GHC (the glasgow haskell compiler). Haskell-Python is written in RPython,
and is compiled into a JIT compiler using the RPython toolchain. Our goal for this
project is to extend Haskell-Python to use GHC as a frontend for compilation 
of Haskell programs.

\subsection{GHC}

GHC (Glasgow Haskell Compiler) is a compiler for the Haskell programming language.
\emph{Haskell} is a \emph{strongly-typed non-strict purely-functional} 
programming language, it will not be described in any detail here, since 
Haskell is not the language we focus on. See \cite{hudak1992report}
for an introduction to Haskell. 

Core is an intermediate language used by the Glasgow Haskell Compiler\cite{ghc},
and it is this language we wish to interpret. Core is a desugared version of Haskell, 
things like pattern matching
and list comprehensions are transformed out to simpler constructs.\cite{jones1994compilation}

\subsection{Extcore}

Extcore is a Haskell package for working with GHC's Core language. Among other things,
it implements a parser for External-core, this is the part used from extcore in this project.

\begin{comment}
\subsubsection{Compilation by transformation}

GHC uses a compilation idiom called \emph{compilation by transformation}. The idea is to repeatedly perform 
correctness-preserving transformations to the program. Ideally, these transformations will result
in a semantically equal program that executes more quickly or in less space. Such transformations seem
to fall into two categories:

\begin{itemize} 
\item{\emph{Glamorous transformations}} are global, sophisticated, intellectually satisfying transformations,
sometimes guided by some interesting kind of analyzis.
\item{\emph{Humble transformations}} are small, simple, local transformations. Individually they look very trivial.
\end{itemize}

In the Glasgow Haskell Compiler, all humble transformations are done by the \emph{simplifier}. 
\cite{jones1994compilation} The simplifier repeatedly performs transformations on the Core language.


\subsection{Extcore}

Extcore is a package for working with the External-core format.

\end{comment}


% Haskell-Python

\section{Haskell-Python}

\subsection{Implemented classes}

\begin{itemize}

\item Symbol: 			 A cached symbol that can be compared by identity (which is not true for strings). 
\item HaskellObject: 		 Base class for all objects that the interpreter handles. 
\item Value:			 Base class for evaluated values (i.e. already in head-normal form). 
\item Constructor:		 A constructor. This is an abstract base class, there are subclasses generated below for various numbers of arguments.
\item ConstructorN:		 
\item AbstractFunction:	 
\item Function:		 	 A user-defined function, i.e. written in Haskell 
\item Rule:			 One rule of a user-defined function. 
\item Substitution:		 The body of a function with numbered variables substituted by values. 
\item PrimFunction:		 A primitive function, i.e. one not implemented in Haskell but at the machine level. 
\item Var:			 A variable.
\item NumberedVar:		 
\item Application:		 A function application. This is an abstract base class, there are subclasses generated below for various numbers of arguments. 
\item ApplicationN:		 
\item Thunk:			 An unevaluated function application. 
\item StackElement:		 Base class of the stack elements of the evaluation stack. 
\item CopyStackElement:	 	Need to copy the top of the stack. 
\item UpdateStackElement:	 Need to update the thunk stored in this after its content has been evaluated. 

\end{itemize}

\subsection{Implemented functions}

The interpreter implements the following functions:

\begin{itemize}

\item make\_arg\_subclass(n, base)
\item make\_constructor(function, args)
\item make\_constr(name, *args)
\item enum(rule, subst)
\item function(name, rules, recursive=False): Takes a function name, and a set of paramters (rules), to generate the Rules object necessary to create a Function object. Returns a Function object.
\item make\_application(function, args)
\item get\_printable\_location(function)
\item evaluate\_hnf(obj): Wrapper for main\_loop, with assertion. Used for testing.
\item main\_loop(expr): Takes the main \emph{function application} as argument, and evaluates
the program

\end{itemize}




% Core

%\clearpage

\section{External-core}

The Core language is an intermediate language used by GHC.
All syntactic sugar is removed, type checking is performed, pattern matching is
translated into case-expressions (each of witch performs only a single level of
matching) and overloading is resolved.\cite{jones1992implementing} Following
is a description of external-core as presented by \cite{tolmach2010ghc}.

\subsection{External-core language definition}

We use the following semantics to define the Core grammar:

\begin{longtable}{ l c l }

$[$ pat $]$		& :	& optional			\\
$\{$ pat $\}$		& :	& zero or more repetitions	\\
$\{$ pat $\}^{+}$	& :	& one or more repetitions	\\
$pat_{1}|pat_{2}$	& :	& choice			\\

\end{longtable}

\begin{footnotesize}
\begin{longtable}{ r c l r }


\\[0.01in]

\multicolumn{4}{l}{Module}			 \\
\\[0.01in]
$module$	& $ \rightarrow $ 	& \%module $mident$ $\{$ $tdefg$ ; $\}$ $\{$ $vdefg$ ; $\}$				&			\\
\\[0.01in]

\multicolumn{4}{l}{Type defn.}			 \\
\\[0.01in]
$tdefg$ 	& $ \rightarrow $	& \%data $qtycon$ $\{$ $tbind$ $\}$  = $\{$ $[$ $cdef$ $\{$ ; $cdef$ $\}$ $]$ $\}$	& algebraic type	\\
		& $ | $			& \%newtype $qtycon$ $qtycon$ $\{ tbind \}$ = $ty$					& newtype		\\
\\[0.01in]

\multicolumn{4}{l}{Constr. defn.}			 \\
\\[0.01in]
$cdef$		& $ \rightarrow $	& $qdcon$ $\{$ @ $tbind$ $\}$ $\{$ $aty$ $\}^{+}$ 					& 			\\
\\[0.01in]

\multicolumn{4}{l}{Value defn.}			 \\
\\[0.01in]
$vdefg$		& $ \rightarrow $	& \%rec $\{$ $vdef$ $\{$ ; $vdef$ $\}$ $\}$						& recursive		\\
		& $ | $			& $vdef$										& non-recursive		\\
$vdef$ 		& $ \rightarrow $	& $qvar$ :: $ty$ = $exp$								& 			\\
\\[0.01in]

\multicolumn{4}{l}{Atomic expr.}			 \\
\\[0.01in]
$aexp$		& $ \rightarrow $	& $qvar$										& variable		\\
		& $ | $			& $qdcon$										& data constructor	\\
		& $ | $			& $lit$											& literal		\\
		& $ | $			& ( $exp$ ) 										& nested expr.		\\
\\[0.01in]

\multicolumn{4}{l}{Expression}			 \\
\\[0.01in]
$exp$		& $ \rightarrow $	& $aexp$										& atomic expr.		\\
		& $ | $			& $aexp$ $\{$ $arg$ $\}^{+}$ 								& application		\\
		& $ | $			& $\backslash$ $\{$ $binder$ $\}$ -$>$ $exp$						& abstraction		\\
		& $ | $			& \%let	$vdefg$ \%in $exp$								& local definition	\\
		& $ | $			& \%case ( $aty$ ) $exp$ \%of $vbind$ $\{$ $alt$ $\{$ ; $alt$ $\}$ $\}$			& case expr.		\\
		& $ | $			& \%cast $exp$ $aty$									& type coercion		\\
		& $ | $			& \%note "  $\{$ $char$ $\}$ " $exp$							& expression note	\\
		& $ | $			& \%external ccal " $\{$ $char$ $\}$ " $aty$						& external reference	\\
		& $ | $			& \%dynexternal ccal $aty$								& external reference (dynamic)	\\
		& $ | $			& \%label " $\{$ $char$ $\}$ "								& external label	\\
\\[0.01in]

\multicolumn{4}{l}{Argument}			 \\
\\[0.01in]
$arg$		& $ \rightarrow $	& @ $aty$										& type argument		\\
		& $ | $			& $aexp$										& value argument	\\
\\[0.01in]

\multicolumn{4}{l}{Case alt}			 \\
\\[0.01in]
$alt$		& $ \rightarrow $	& $qdcon$ $\{$ @ $tbind$ $\}$ $\{$ $vbind$ $\}$ -$>$ $exp$				& constructor alternative \\
		& $ | $			& $lit$ -$>$ $exp$									& literal alternative 	\\
		& $ | $			& \%\_ -$>$ $exp$									& default alternative	\\
\\[0.01in]

\multicolumn{4}{l}{Binder}			 \\
\\[0.01in]
$binder$	& $ \rightarrow $	& @ $tbind$										& type binder		\\
		& $ | $			& $vbind$										& value binder		\\
\\[0.01in]

\multicolumn{4}{l}{Type binder}			 \\
\\[0.01in]
$tbind$		& $ \rightarrow $	& $tyvar$										& implicit of kind * 	\\
		& $ | $			& ( $tyvar$ :: $kind$ )									& explicitly kinded	\\
\\[0.01in]

\multicolumn{4}{l}{Value binder}			 \\
\\[0.01in]
$vbind$		& $ \rightarrow $	& ( $var$ :: $ty$ )									& \\
\\[0.01in]

\multicolumn{4}{l}{Literal}			 \\
\\[0.01in]
$lit$		& $ \rightarrow $	& ( $[$-$]$ $\{$ $digit$ $\}^{+}$ :: $ty$ )						& integer 		\\ 
		& $ | $			& ( $[$-$]$ $\{$ $digit$ $\}^{+}$ \% $\{$ $digit$ $\}^{+}$ :: $ty$ )			& rational		\\
		& $ | $			& ( ' $char$ ' :: $ty$ )								& character		\\
		& $ | $			& ( " $\{$ $char$ $\}$ " :: $ty$ )							& string		\\
\\[0.01in]

\multicolumn{4}{l}{Character}			 \\
\\[0.01in]
$char$		& $ \rightarrow $	& \multicolumn{2}{l}{Any ASCII character in range 0x20-0x7E except 0x22, 0x27, 0x5c}			 \\
		& $ | $			& $\backslash$x $hex$ $hex$								& ASCII code escape sequence \\
$hex$		& $ \rightarrow $	& 0 $|$ ... $|$ 9 $|$ a $|$ ... f							& \\
\\[0.01in]

\multicolumn{4}{l}{Atomic type}			 \\
\\[0.01in]
$aty$		& $ \rightarrow $	& $tyvar$										& type variable 	\\
		& $ | $			& $qtycon$										& type constructor	\\
		& $ | $ 		& ( $ty$ )										& nested type 		\\
\\[0.01in]

\multicolumn{4}{l}{Basic type}			 \\
\\[0.01in]
$bty$		& $ \rightarrow $	& $aty$											& atomic type		\\
		& $ | $			& $bty$ $aty$										& type application	\\
		& $ | $			& \%trans $aty$ $aty$									& transitive coercion 	\\
		& $ | $			& \%sym	$aty$										& symmetric coercion	\\
		& $ | $			& \%unsafe $aty$ $aty$									& unsafe coercion	\\
		& $ | $			& \%left $aty$										& left coercion		\\
		& $ | $			& \%right $aty$										& right coercion	\\
		& $ | $			& \%inst $aty$ $aty$									& instantiation coercion \\
\\[0.01in]

\multicolumn{4}{l}{Type}			 \\
\\[0.01in]
$ty$		& $ \rightarrow $	& $bty$											& basic type 		\\
		& $ | $			& \%forall $\{$ $tbind$ $\}^{+}$ . $ty$							& type abstraction	\\
		& $ | $			& $bty$ -$>$ $ty$									& arrow type construction \\
\\[0.01in]

\multicolumn{4}{l}{Atomic kind}			 \\
\\[0.01in]
$akind$		& $ \rightarrow $	& $*$											& lifted kind 		\\
		& $ | $			& \#											& unlifted kind 	\\
		& $ | $			& ?											& open kind 		\\
		& $ | $			& $bty$ :=: $bty$									& equality kind 	\\
		& $ | $			& ( $kind$ ) 										& nested kind 		\\
\\[0.01in]

\multicolumn{4}{l}{Kind}			 \\
\\[0.01in]
$kind$		& $ \rightarrow $	& $akind$										& atomic kind		\\
		& $ | $			& $akind$ -$>$ $kind$									& arrow kind		\\
\\[0.01in]

\multicolumn{4}{l}{Identifier}			 \\
\\[0.01in]
$mident$	& $ \rightarrow $	& $pname$ : $uname$									& module		\\
$tycon$		& $ \rightarrow $	& $uname$										& type constr.		\\
$qtycon$	& $ \rightarrow $	& $mident$ . $tycon$									& qualified type constr.\\
$tyvar$		& $ \rightarrow $	& $lname$										& type variable		\\
$dcon$		& $ \rightarrow $	& $uname$										& data constr.		\\
$qdcon$		& $ \rightarrow $	& $mident$ . $dcon$									& qualified data constr.\\
$var$		& $ \rightarrow $	& $lname$										& variable		\\
$qvar$		& $ \rightarrow $	& $[$ $mident$ . $]$ $var$								& optionally qualified variable\\
\\[0.01in]

\multicolumn{4}{l}{Name}			 \\
\\[0.01in]
$lname$		& $ \rightarrow $	& $lower$ $\{$ $namechar$ $\}$								& \\
$uname$		& $ \rightarrow $	& $upper$ $\{$ $namechar$ $\}$								& \\
$pname$		& $ \rightarrow $	& $\{$ $namechar$ $\}^{+}$								& \\
$namechar$	& $ \rightarrow $	& $lower$ $|$ $upper$ $|$ $digit$							& \\
$lower$		& $ \rightarrow $	& a$|$b$|$...$|$z$|$\_									& \\
$upper$		& $ \rightarrow $	& A$|$B$|$...$|$Z$|$									& \\
$digit$		& $ \rightarrow $	& 0$|$1$|$...$|$9									& \\
\\[0.01in]

\end{longtable}
\end{footnotesize}



\subsection{Informal semantics of Core}

Core resembles a explicitly-typed polymorphic lambda-calculus ($F_{w}$), with some additions,
local let bindings, algebraic type definitions, constructors, case-expressions, primitive types,
literals and operators.\cite{tolmach2010ghc}

\subsubsection{Program organization and modules}

Programs represented in Core are organized into modules corresponding directly to source-level
Haskell modules. A module identifier (\emph{mident}) consists of a \emph{package name} followed
by a module name. 

Each module may contain each of the following top-level declarations:
\begin{itemize}
\item{Algebraic datatype declarations:} each defining a type constructor and one or more data
constructors.
\item{Newtype declarations:} corresponding to Haskell newtype declarations, each defining a 
type constructor and a coercion name.
\item{Value declarations:} defining the types and values of top-level variables.
\end{itemize}


\cite{tolmach2010ghc}


\subsubsection{Namespaces}

There are five distinct namespaces:
\begin{enumerate}

\item module identifiers (\emph{mident})
\item type constructors (\emph{tycon})
\item type variables (\emph{tyvar})
\item data constructors (\emph{dcon})
\item term variables (\emph{var})
\end{enumerate}

A variable (type or term) may have multiple definitions within a module. However, they
never shadow one another, the scope of the definition of a variable never contain a
redefinition of the same variable. Type variables may be "shadowed". Thus if a variable has
multiple definitions, they must be local (let-bound).\cite{tolmach2010ghc}

\subsubsection{Types and Kinds}

\paragraph{Types}

\paragraph{Coercions}

\paragraph{Kinds}

\paragraph{Lifted and unlifted types}

\paragraph{Type constructors; base kinds and higher kinds}

\paragraph{Type synonyms and type equivalence}

\subsubsection{Algebraic data types}


\subsubsection{Newtypes}


\subsubsection{Expression forms}


\subsubsection{Expression evaluation}


\subsection{Primitive module}



% JSON Core

%\clearpage

\section{JSON representation of Core}

JavaScript Object Notation (JSON) is a lightweight data interchange format.

Since a library for manipulating JSON is available for Haskell, this
makes it a good choice for the project. In addition, it is easy to parse and
the Haskell library contains a pretty-printer, making the result easier to
inspect.

\clearpage

\subsection{Formal definition of JSON}

\begin{scriptsize}
\begin{longtable}{ r c l r }
\multicolumn{4}{l}{Object}		\\
\\[0.01in]
$object$	& $ \rightarrow $ 	& \{ \}					& \\
 		& $ | $			& \{ $members$ \} 			& \\
$members$ 	& $ \rightarrow $	& $pair$				& \\
		& $ | $			& $pair$ , $members$ 			& \\
$pair$		& $ \rightarrow $	& $string$ : $value$ 			& \\
\\[0.01in]

\multicolumn{4}{l}{Array}		\\
$array$		& $ \rightarrow $	& [ ]					& \\
		& $ | $			& [ $elements$ ]			& \\
$elements$ 	& $ \rightarrow $	& $value$				& \\
		& $ | $			& $value$ , $elements$			& \\
\\[0.01in]

\multicolumn{4}{l}{Value}		\\
$value$		& $ \rightarrow $	& $string$				& \\
		& $ | $			& $number$				& \\
		& $ | $			& $object$				& \\
		& $ | $			& $array$				& \\
		& $ | $			& true					& \\
		& $ | $			& false					& \\
		& $ | $			& null					& \\
\\[0.01in]

\multicolumn{4}{l}{String}		\\
$string$	& $ \rightarrow $	& ""					& \\
		& $ | $			& " $chars$ "				& \\
$chars$		& $ \rightarrow $	& $char$				& \\
		& $ | $			& $char$ $chars$			& \\
$char$		& $ \rightarrow $	& any Unicode character except $"$ 	& \\ 
		&			& or $\backslash$ or control characters: & \\
		&			& $\backslash\backslash$		& \\
		&			& $\backslash /$ 			& \\
		&			& $\backslash b$ 			& \\
		& 			& $\backslash f$ 			& \\
		&			& $\backslash n$			& \\
		& 			& $\backslash r$ 			& \\
		&			& $\backslash t$ 			& \\
		& 			& $\backslash u$ four-hex digits\\
\\[0.01in]

\multicolumn{4}{l}{Number}		\\
$number$	& $ \rightarrow $ 	& $int$ 				& \\
		& $ | $			& $int$ $frac$				& \\
		& $ | $			& $int$ $exp$				& \\
		& $ | $			& $int$ $frac$ $exp$			& \\
$int$		& $ \rightarrow$ 	& $digit$				& \\
		& $ | $ 		& $digit1-9$ $digits$			& \\
		& $ | $ 		& - $digit$				& \\
		& $ | $ 		& - $digit1-9$ $digits$			& \\
$frac$ 		& $ \rightarrow $ 	& . $digits$ 				& \\
$exp$		& $ \rightarrow $ 	& $e$ $digits$ 				& \\
$digits$	& $ \rightarrow $ 	& $digit$				& \\
		& $ | $ 		& $digit$ $digits$			& \\
$e$		& $ \rightarrow $ 	& e					& \\
		& $ | $ 		& e+					& \\
		& $ | $ 		& e- 					& \\
		& $ | $ 		& E					& \\
		& $ | $ 		& E+					& \\
		& $ | $ 		& E-					& \\
\\[0.01in]

\caption{Grammar for JSON}
\label{json}
\end{longtable}

\end{scriptsize}

\clearpage

\subsection{JSON representation of Core}

In order to work with JSON and external-core, a format was defined that
expresses the Core program in JSON notation. Most of the right hand side of 
the grammar evaluates to JSON Values. 
Even though the grammar is changed to support JSON, an effort was made to
keep it similar to the original Core grammar for easy referencing. The size
of the resulting files was not considered to be an issue.

The following definitions was used to describe the grammar:

\clearpage

\begin{scriptsize}
\begin{longtable}{ c c l }


$[$ $pat$ $]$ 		& : 	& Zero or more repetitions of $pat$ surrounded by $[$ $]$ and comma separated (A JSON Aray). 	\\
$[$ $pat$ $]^{+}$ 	& : 	& One or more repetitions of $pat$ surrounded by $[$ $]$ and comma separated (A JSON Aray). 	\\ 
$\{$ $pat$ $\}$		& :	& Represents a JSON Object, $pat$ is a JSON $members$.						\\
$pat_{1}$ $|$ $pat_{2}$	& :	& Choice.											\\
$||$ $pat$ $||$ 	& :	& Optional											\\
\\[0.01in]

\end{longtable}
\end{scriptsize}

JSON Core grammar:

\begin{scriptsize}
\begin{longtable}{ r c l r }

\\[0.01in]
\multicolumn{4}{l}{Module}		\\
$module$	& $ \rightarrow $ 	& $\{$ "\%module" : $mident$ , "tdefg" : $[$ $tdefg$ $]$ , "vdefg" : $[$ $vdefg$ $]$ $\}$			&			\\
\\[0.01in]

\multicolumn{4}{l}{Type defn.}		\\
$tdefg$ 	& $ \rightarrow $	& $\{$ "\%data" : $qtycon$ , "tbind" : $[$ $tbind$ $]$, "cdef" : $[$ $cdef$ $]$ $\}$							& algebraic type	\\
		& $ | $			& $\{$ "\%newtype" : $qtycon$ , "qtycon" : $qtycon$ , "tbind" : $[$ $tbind$ $]$ , "ty" : $ty$ $\}$ 		& newtype		\\
\\[0.01in]

\multicolumn{4}{l}{Constr. defn.}	\\
\\[0.01in]
$cdef$		& $ \rightarrow $	& $\{$ "qdcon" : $qdcon$ , "tbind" : $[$ $tbind$ $ ]$ , "aty" : $[$aty$]^{+}$ $\}$ 				& 			\\
\\[0.01in]

\multicolumn{4}{l}{Value defn.}		\\
\\[0.01in]
$vdefg$		& $ \rightarrow $	& $\{$ "\%rec" : $[$ $vdef$ $]^{+}$ $\}$    									& recursive		\\
		& $ | $			& $vdef$													& non-recursive		\\
$vdef$ 		& $ \rightarrow $	& $\{$ "qvar" : $qvar$ , "ty" : $ty$ , "exp" : $exp$ $\}$ 							& 			\\
\\[0.01in]

\multicolumn{4}{l}{Atomic expr.}	\\
\\[0.01in]
$aexp$		& $ \rightarrow $	& $qvar$													& variable		\\
		& $ | $			& $qdcon$													& data constructor	\\
		& $ | $			& $lit$														& literal		\\
		& $ | $			& $\{$ "exp" : $exp$ $\}$ 											& nested expr.		\\
\\[0.01in]

\multicolumn{4}{l}{Expression}			 \\
\\[0.01in]
$exp$		& $ \rightarrow $	& $aexp$													& atomic expr.		\\
		& $ | $			& $\{$ "aexp" : $aexp$ , "args" : $[$ $arg$ $]^{+}$ $\}$ 							& application		\\
		& $ | $			& $\{$ "lambda" : $[$ $binder$ $]$ , "exp" : $exp$ $\}$								& abstraction		\\
		& $ | $			& $\{$ "\%let" : $vdefg$ , "\%in" : $exp$ $\}$									& local definition	\\
		& $ | $			& $\{$ "\%case" : $aty$ , "exp" : $exp$ , "\%of" : $vbind$, "alt" : $[$ $alt$ $]^{+}$ $\}$			& case expr.		\\
		& $ | $			& $\{$ "\%cast" : $exp$ , "aty" : $aty$	$\}$									& type coercion		\\
		& $ | $			& $\{$ "\%note" : "  $\{$ $char$ $\}$ " , "exp" : $exp$	$\}$							& expression note	\\
		& $ | $			& $\{$ "\%external ccal" : " $\{$ $char$ $\}$ " , "aty" : $aty$ $\}$						& external reference	\\
		& $ | $			& $\{$ "\%dynexternal ccal" : $aty$ $\}$									& external reference (dynamic)	\\
		& $ | $			& $\{$ "\%label" : " $\{$ $char$ $\}$ " $\}$									& external label	\\
\\[0.01in]

\multicolumn{4}{l}{Argument}			 \\
\\[0.01in]
$arg$		& $ \rightarrow $	& $\{$ "aty" : $aty$ $\}$											& type argument		\\
		& $ | $			& $\{$ "aexp" : $aexp$ $\}$											& value argument	\\
\\[0.01in]

\multicolumn{4}{l}{Case alt}			 \\
\\[0.01in]
$alt$		& $ \rightarrow $	& $\{$ "qdcon" : $qdcon$ , "tbind" : $[$ $tbind$ $]$ , "vbind" : $[$ $vbind$ $]$ , "exp" : $exp$ $\}$		& constructor alternative \\
		& $ | $			& $\{$ "lit" : $lit$ , "exp" : $exp$ $\}$									& literal alternative 	\\
		& $ | $			& $\{$ "\%\_" : $exp$ $\}$											& default alternative	\\
\\[0.01in]

\multicolumn{4}{l}{Binder}			 \\
\\[0.01in]
$binder$	& $ \rightarrow $	& $\{$ "tbind" : $tbind$ $\}$											& type binder		\\
		& $ | $			& $\{$ "vbind" : $vbind$ $\}$											& value binder		\\
\\[0.01in]

\multicolumn{4}{l}{Type binder}			 \\
\\[0.01in]
$tbind$		& $ \rightarrow $	& $\{$ "tyvar" : $tyvar$ $\}$											& implicit of kind * 	\\
		& $ | $			& $\{$ "tyvar" : $tyvar$ , "kind" : $kind$ $\}$									& explicitly kinded	\\
\\[0.01in]

\multicolumn{4}{l}{Value binder}			 \\
\\[0.01in]
$vbind$		& $ \rightarrow $	& $\{$ "var" : $var$ , "ty" $ty$ $\}$ 										& \\
\\[0.01in]

\multicolumn{4}{l}{Literal}			 \\
\\[0.01in]
$lit$		& $ \rightarrow $	& $jsstring$													& string 		\\ 
		& $ | $			& $jsnumber$													& number		\\
\\[0.01in]

\multicolumn{4}{l}{JSON String}			 \\
\\[0.01in]
$jsstring$	& $ \rightarrow $	& ""														& \\
		& $ | $			& " $jschars$ "													& \\
$jschars$	& $ \rightarrow $	& $jschar$													& \\
		& $ | $			& $jschar$ $jschars$												& \\
$jschar$	& $ \rightarrow $	& any Unicode character except $"$ 										& \\ 
		&			& or $\backslash$ or control characters: 									& \\
		&			& $\backslash\backslash$											& \\
		&			& $\backslash /$ 												& \\
		&			& $\backslash b$ 												& \\
		& 			& $\backslash f$ 												& \\
		&			& $\backslash n$												& \\
		& 			& $\backslash r$ 												& \\
		&			& $\backslash t$ 												& \\
		& 			& $\backslash u$ four-hex digits\\
\\[0.01in]

\multicolumn{4}{l}{JSON Number}			 \\
\\[0.01in]
$jsnumber$	& $ \rightarrow $ 	& $jsint$ 													& \\
		& $ | $			& $jsint$ $jsfrac$												& \\
		& $ | $			& $jsint$ $jsexp$												& \\
		& $ | $			& $jsint$ $jsfrac$ $jsexp$											& \\
$jsint$		& $ \rightarrow$ 	& $jsdigit$													& \\
		& $ | $ 		& $jsdigit1-9$ $jsdigits$											& \\
		& $ | $ 		& - $jsdigit$													& \\
		& $ | $ 		& - $jsdigit1-9$ $jsdigits$											& \\
$jsfrac$ 	& $ \rightarrow $ 	& . $jsdigits$ 													& \\
$jsexp$		& $ \rightarrow $ 	& $jse$ $jsdigits$ 												& \\
$jsdigits$	& $ \rightarrow $ 	& $jsdigit$													& \\
		& $ | $ 		& $jsdigit$ $jsdigits$												& \\
$jse$		& $ \rightarrow $ 	& e														& \\
		& $ | $ 		& e+														& \\
		& $ | $ 		& e- 														& \\
		& $ | $ 		& E														& \\
		& $ | $ 		& E+														& \\
		& $ | $ 		& E-														& \\
\\[0.01in]


\multicolumn{4}{l}{Atomic type}			 \\
\\[0.01in]
$aty$		& $ \rightarrow $	& $\{$ "tyvar" : $tyvar$ $\}$											& type variable 	\\
		& $ | $			& $\{$ "qtycon" : $qtycon$ $\}$											& type constructor	\\
		& $ | $ 		& $\{$ "ty" : $ty$ $\}$												& nested type 		\\
\\[0.01in]

\multicolumn{4}{l}{Basic type}			 \\
\\[0.01in]
$bty$		& $ \rightarrow $	& $aty$														& atomic type		\\
		& $ | $			& $\{$ "bty" : $bty$ , "aty" , $aty$ $\}$									& type application	\\
		& $ | $			& $\{$ "\%trans" : $aty$ , "aty" : $aty$ $\}$									& transitive coercion 	\\
		& $ | $			& $\{$ "\%sym" : $aty$ $\}$											& symmetric coercion	\\
		& $ | $			& $\{$ "\%unsafe" : $aty$ , "aty" : $aty$ $\}$									& unsafe coercion	\\
		& $ | $			& $\{$ "\%left" : $aty$ $\}$											& left coercion		\\
		& $ | $			& $\{$ "\%right" : $aty$ $\}$											& right coercion	\\
		& $ | $			& $\{$ "\%inst" : $aty$ , "aty" : $aty$ $\}$									& instantiation coercion \\
\\[0.01in]

\multicolumn{4}{l}{Type}			 \\
\\[0.01in]
$ty$		& $ \rightarrow $	& $bty$														& basic type 		\\
		& $ | $			& $\{$ "\%forall" :  $[$ $tbind$ $]^{+}$ , "ty" : $ty$ $\}$							& type abstraction	\\
		& $ | $			& $\{$ "bty" $bty$ , "ty" : $ty$ $\}$										& arrow type construction \\
\\[0.01in]

\multicolumn{4}{l}{Atomic kind}			 \\
\\[0.01in]
$akind$		& $ \rightarrow $	& $*$														& lifted kind 		\\
		& $ | $			& \#														& unlifted kind 	\\
		& $ | $			& ?														& open kind 		\\
		& $ | $			& $\{$ "bty" : $bty$ , "bty" : $bty$ $\}$									& equality kind 	\\
		& $ | $			& $\{$ "kind" : $kind$ $\}$ 											& nested kind 		\\
\\[0.01in]

\multicolumn{4}{l}{Kind}			 \\
\\[0.01in]
$kind$		& $ \rightarrow $	& $\{$ "akind" : $akind$ $\}$											& atomic kind		\\
		& $ | $			& $\{$ "akind" : $akind$ , "kind" : $kind$ $\}$									& arrow kind		\\
\\[0.01in]

\multicolumn{4}{l}{Identifier}			 \\
\\[0.01in]
$mident$	& $ \rightarrow $	& " $pname$ : $uname$ "												& module		\\
$tycon$		& $ \rightarrow $	& " $uname$ "													& type constr.		\\
$qtycon$	& $ \rightarrow $	& " $mident$ . $tycon$ "											& qualified type constr.\\
$tyvar$		& $ \rightarrow $	& " $lname$ "													& type variable		\\
$dcon$		& $ \rightarrow $	& " $uname$ "													& data constr.		\\
$qdcon$		& $ \rightarrow $	& " $mident$ . $dcon$ "												& qualified data constr.\\
$var$		& $ \rightarrow $	& " $lname$ "													& variable		\\
$qvar$		& $ \rightarrow $	& " $||$ $mident$ . $||$ $var$ "										& optionally qualified variable\\
\\[0.01in]

\multicolumn{4}{l}{Name}			 \\
\\[0.01in]
$lname$		& $ \rightarrow $	& $lower$ $\{$ $namechar$ $\}$								& \\
$uname$		& $ \rightarrow $	& $upper$ $\{$ $namechar$ $\}$								& \\
$pname$		& $ \rightarrow $	& $\{$ $namechar$ $\}^{+}$								& \\
$namechar$	& $ \rightarrow $	& $lower$ $|$ $upper$ $|$ $digit$							& \\
$lower$		& $ \rightarrow $	& a$|$b$|$...$|$z$|$\_									& \\
$upper$		& $ \rightarrow $	& A$|$B$|$...$|$Z$|$									& \\
$digit$		& $ \rightarrow $	& 0$|$1$|$...$|$9									& \\
\\[0.01in]

\caption{Grammar for JSCore}
\label{jscore}

\end{longtable}
\end{scriptsize}

\clearpage



% Implementation

%\clearpage

\section{Implementation}

\subsection{Tools and versions}

The following tools and package versions was used in the implementation:

\begin{itemize}
\item GHC version 7.0.3
\item extcore version 1.0.1
\item PyPy current head branch (Last tested 09/11/2011)
\item Haskell-Python interpreter: Interpreter of Core' written in RPython
\item Python 2.7: Used to test the Core' interpreter without it having to be correct RPython.
(pypy-python could also have been used)
\end{itemize}

%\paragraph{GHC} version 7.0.3. Binary version. % Source needed for creating extcore with correct grammar file.

%\paragraph{genprimopcode} --make-ext-core-source < \{path/to/primops.txt\} > \{path/to/PrimEnv.hs\}

%\paragraph{extcore} version 1.0.1, %Source version, updated with grammar of external-core for GHC version 7.0.3 (See extcore 1.0.1 README file for details)

%\paragraph{PYPY} current head brach.

%\paragraph{Haskell-Python} interpreter of core written in RPython.


\subsection{Pipeline}

The project implements the following pipeline (see figure \ref{core2js}):
\begin{enumerate}
\item Serialize Haskell program: 
  \begin{enumerate}
  \item Create external-core file from Haskell program using GHC.
  \item Create JSCore from external-core using the extcore and JSON packages.
  \end{enumerate}
\item Deserialize JSCore:
  \begin{enumerate}
  \item Parse JSCore using the parsing tools available for PyPy
  \item Build Core AST from resulting JSON datatype
  \end{enumerate}
\item Evaluate program:
  \begin{enumerate}
  \item Evaluation is done by the already implemented PyPy Core' interpreter,
  Haskell-Python. Additional functionality had to be built on top of this, mostly 
  Haskell library functions.
  \end {enumerate}
\end{enumerate}

\subsection{Organization}

The implementation is organized as represented by figure \ref{organization}. The
main folder (interpreter) contains the main program, and a program for generating
dot files, "makegraph.py" (used to create graphs of parsed JSCore files using graphviz). 
The "haskell" folder
contains the PyPy Core' interpreter code, used by the main program for evaluation, 
and by the parser to generate the abstract-syntax-tree (AST). In addition to this,
the subfolder packages implements some simple functionality to be used by the test-programs.
Among others, a very simple IO function for printing text to the terminal (putStrLn).
These packages are loaded and references to the functions they contain are used during
the creation of the AST. See figure \ref{core2js} for a simple description of the pipeline.

The "core" folder contains a Haskell program for generating JSCore files from external-core
files, the JSCore parser, and a simple datastructure representing Haskell modules.

\begin{figure}[H]
\centering
\includegraphics[width=0.8\textwidth]{diags/organization}
\caption{Code tree: The top of the boxes is the folder name, and the rest is the source 
files. Arrows represent subfolders.}
\label{organization}
\end{figure}

\subsection{Serializer}

The serializer consists of two parts; GHC generating external-core, and 
a Haskell program to generate JSCore (core2js).

External-core is easily generated by using a compiler flag:
\begin{lstlisting}
ghc -fext-core {path-to-program}
\end{lstlisting}

\begin{figure}[H]
\centering
\includegraphics[width=0.8\textwidth]{diags/pipe_w_core2js}
\caption{Pipeline implementation with core2js using extcore}
\label{core2js}
\end{figure}

The Haskell program generating JSCore uses the extcore packages. This package
implements functionality for working with external-core. The result is a datastructure
mapping directly to the external-core format defined in \cite{tolmach2010ghc}. By traversing
this structure, the program builds up a JSON tree using the Haskell JSON package.
The result is a tree of JSON constructs (corresponding to the grammar defined in table \ref{jscore}), 
this is then pretty-printed and dumped to a file.

\subsection{Deserializer}

The deserializer implements a JSON parser. The resulting datastructure is then traversed, building up
an AST using the constructs defined in the Haskell-Python interpreter. 

PyPy implements a parser generator, this simply takes a grammar defined as a string, written in
extended-backus-naur-form (EBNF), and generates a parser. This parser is then used to create a 
JSON datastructure, as represented by table \ref{json}.
The resulting datastructure is then traversed. By checking the contents of the JSON constructs
with the actual external-core format, the Core AST is built. External functionality is imported
from the Python implementations of the Haskell libraries as it is encountered. 

After this is done, we are left with a "module" object, corresponding to the initial Haskell
module. 

\subsection{Haskell libraries}

To make some simple test-cases work, some basic Haskell functionality had to be implemented.
Some of this functionality was implemented already in the Haskell-Python Core' interpreter.
The work done here was mostly to organize the functionality into modules corresponding
to Haskell modules. The functionality implemented in these modules does however, not correspond
to the Haskell implementations. This is left for future work, as this is a large task.

From figure \ref{fig:helloworldgraph} (representing a "hello world" program in JSCore), the atomic 
expression "base:SystemziIO.putStrLn" corresponds
to the Haskell function "putStrLn", which is located in the Haskell module "System.IO". This is translated
into a reference to the function "putStrLn" defined in the python module located in 
"haskell/packages/System/IO.py". See figure \ref{organization}.

\subsection{Evaluation}

In order to evaluate the Haskell programs correctly, the expression "main:ZCMain.main" would have
to be reduced to \emph{WHNF}. However, this would require a lot of the functionality used by GHC
to be implemented. Specifically, "GHC.TopHandler.runMainIO()". In order to implement this function
a lot of other functionality would have to be implemented. The function is a wrapper around 
"main:Main.main", it catches uncaught exceptions and flushes stdout/stderr before exiting. 
Implementing this is a goal for further
development, but currently a simple hack is to only evaluate the expression "main:Main.main". This way,
simple programs can be tested by implementing the necessary functionality at a high level, such as the
"putStrLn" function which is implemented as a simple "print" Python function.

\subsection{Issues}

\begin{figure}[H]
\centering
\includegraphics[width=0.8\textwidth]{diags/pipe_w_haskell2js}
\caption{Pipeline with implementation of haskell2js using the GHC API}
\label{haskell2js}
\end{figure}

Very few of the initial plans regarding this project was actually realized in the implementation.
The GHC API was intended to be used to generate JSCore, but this failed do to a lack of experience
with Haskell and the GHC API (see figure \ref{haskell2js} for a description of the initially intended
pipeline). It was then discovered that a lot of the necessary code was already
written in GHC (the code that generates the external-core files), however, this code was not 
exported in any way. It was also deeply embedded in the GHC code. After several attempts at creating
the JSCore format, and some emails back and forth with the GHC team, these methods where abandoned.

The extcore package was then introduced, as it was thought to serve nicely for our purpose. This
does however add an extra step to the generation of JSCore, having to generate external-core
first. There is no way of working with external-core without parsing it from a file. See figure \ref{core2js}.

The reason for not using extcore from the beginning was that it was thought to not be very 
well supported, as the external-core format seems to be changing. This also turned out to be
the case. However, GHC also changes rapidly, and an implementation using the GHC API may not
work for very long either. A version linking to the GHC executable would most likely be the
worst choice. As this also changes rapidly.

A large amount of time was spent trying to generate this intermediate format. And the
greatest obstacle was that the tools being used was either incompatible, or lacking in
documentation. Either way, this problem will have to be revisited and solved.

The method described here worked well very trivial Haskell programs, however, 
it turned out that
the extcore package was not able to parse any nontrivial external-core files generated 
by the GHC version used. It was thought that this was due to the fact that the extcore
package was written for earlier versions of GHC (6.10 and 6.12). However, these versions
turned out to have the same problem when tested.



% Examples

%\clearpage

\section{Examples}

\lstset{ %
frame=single,                   % adds a frame around the code
tabsize=2,                      % sets default tabsize to 2 spaces
captionpos=b,                   % sets the caption-position to bottom
breaklines=true,                % sets automatic line breaking
}

\subsection{Example 1: hello world}

This example program is a simple "hello world" program, as this was practically
the only program that was able to pass through the entire pipeline. Following
is the "hello world" program written in Haskell:

\begin{footnotesize}
\lstinputlisting[language=Haskell]{"../interpreter/tests/helloworld.hs"}
\end{footnotesize}

\subsubsection{Converted to Core}

After the program has passed through GHC and the external-core file
has been generated, the program looks like this:

\begin{footnotesize}
\lstinputlisting{"../interpreter/tests/helloworld.hcr"}
\end{footnotesize}

... TODO: Explain this representation in more detail.

\subsubsection{Converted to JSCore}

In the next step it is parsed and dumped to JSCore:

\begin{footnotesize}
\lstinputlisting{"../interpreter/tests/helloworld.hcj"}
\end{footnotesize}

\subsubsection{JSCore graph}

Using the parsing libraries of PyPy we can generate a nice graph from the result,
directly corresponding to the resulting datastructure. 
See figure \ref{fig:helloworldgraph}.
By simply traversing this datastructure we can generate the AST for the Core 
interpreter (Haskell-Python).

\begin{sidewaysfigure}
\begin{figure}[H]
\includegraphics[width=\textwidth]{../interpreter/tests/helloworld.png}
\caption{Example program translated to JSON}
\label{fig:helloworldgraph}
\end{figure}
\end{sidewaysfigure}

\subsubsection{Result}

The program results as expected, outputting the string "Hello, world!".

\begin{comment}

\subsection{Example 2: naive fibonacci}

The following program is a simple fibonacci program.

\begin{footnotesize}
\lstinputlisting[language=Haskell]{"../interpreter/tests/fib.hs"}
\end{footnotesize}

\subsubsection{Converted to Core}

As we can see, the simple hello world program becomes more complex when translated
to Core by GHC.

\begin{footnotesize}
\lstinputlisting{"../interpreter/tests/fib.hcr"}
\end{footnotesize}

%TODO: Explain this representation in more detail.

\subsubsection{Converted to JSCore}

And translated to JSCore by our serializer:

\begin{footnotesize}
\lstinputlisting{"../interpreter/tests/fib.hcj"}
\end{footnotesize}

\subsubsection{JSCore graph}

Using the parsing libraries of PyPy we can generate a nice graph from the result, 
see figure \ref{fig:fibgraph}

By simply traversing this datastructure we can generate the AST for the Core interpreter.

\begin{sidewaysfigure}
\begin{figure}[H]
\includegraphics[width=\textwidth]{../interpreter/tests/fib.png}
\caption{Example program 2 translated to JSON}
\label{fig:fibgraph}
\end{figure}
\end{sidewaysfigure}

\subsubsection{Result}

\end{comment}


% Future work

%\clearpage
\section{Future work}

\subsection{Getting programs into Haskell-Python from GHC}

This project focused mainly on this task, which was thought to be simple.
However, due to a combination of reasons, this turned out not to be the case.
Mainly, inexperience; with Haskell, the GHC API, and functional languages in 
general. However, a lot of experience was gained during this project, and
future development will benefit from this.

The work involved in this will be revisiting the different possibilities to
achieve our goal. These options are:

\begin{itemize}
\item Write an external-core parser directly in RPython, using files generated by GHC.
\item Include development of extcore as a part of the project. This most likely a bad
idea, as this seems to be no easier than the alternatives.
\item Use the GHC API to generate JSCore, this is also nontrivial.
\item Create functionality to be linked into the GHC executable, in order to generate the
representation of JSCore. Also nontrivial.
\item Simply use the version of GHC matching the version of extcore. This will not be
a good idea for further development of the project, but may be a simple solution to get
a prototype working quickly.
\item Implement a Haskell pipeline in RPython, including parsing, typechecking and desugaring.
This will be a very large task.
\end{itemize}

\subsection{Rewrite the deserializer to proper RPython}

The deserializer is currently not written in proper RPython. 
Converting the code to RPython will alow it to be compiled to a JIT interpreter by
the RPython toolchain. This should not be a very big task, but it requires understanding
of the RPython coding style. There are also restrictions on how one may use the pypy parser
tools.

\subsection{Implement base Haskell libraries}

Implementing Haskell libraries is necessary to run any Haskell program passed 
through GHC. One option may be to implement (or automatically generate from GHC code) Haskell
primitive types, and to convert the Haskell base libraries to JSCore.

\begin{comment}
\subsection{Map GHC encoded Types to Haskell-Python}

Figure out how to create encoded types for Haskell-Python. It may be possible to
autogenerate these from GHC files.

"2.) understanding how GHC encodes types. The Core Haskell of the previous steps encodes the types of all functions in slightly low-level
ways. This needs to be understood and a mapping of these types to what
the Python Haskell interpreter provides needs to be written." 

\subsection{Set up GHC test environment for Haskell-Python}

Setting up the GHC test enviromnent for Haskell-Python would be very valuable
for development and bug fixing.

"3.) the actual interpretation of the Core language is mostly
implemented. There are probably some things missing, which will be
discovered by running some Haskell programs. For that end, it would be
good to find out whether there is a Haskell implementation test suite
and get it to run."

\subsection{Add built in Haskell types to run some Haskell benchmarks}

"4.) what is missing to run more non-pure Haskell programs are all the
built-in functions (e.g. those that perform arithmetic, I/O, call C
functions, etc) and built-in types (e.g. integers, floats, C-level types
like arrays and structs). These should be added step by step. This is an
essentially open-ended task. It would be good to add as many built-ins
so that some of the Haskell benchmarks can run."

\subsection{Optimize PyPy JIT for Haskell-Python}

"5.) JIT work: While the JIT of PyPy can mostly be automatically applied,
in practice a lot of careful work is needed to make sure that the
generated code is optimal (or at least good). To do that, a test suite
of Haskell snippets that explicitly compares the generated machine code
with what it should look like is needed, and then the careful adding of
some tests to this suite, together with the better placement of JIT
hints. This is both the hardest step, as well as the most exciting one."
\end{comment}


% Appendixes
\appendix

% Core grammar
\clearpage
\appendixpage

\section{Formal definition of External-core}
\label{coregrammar}

The following semantics is used to define the Core grammar, 
as seen in \cite{tolmach2010ghc}:

\begin{longtable}{ l c l }

$[$ pat $]$		& :	& optional			\\
$\{$ pat $\}$		& :	& zero or more repetitions	\\
$\{$ pat $\}^{+}$	& :	& one or more repetitions	\\
$pat_{1}|pat_{2}$	& :	& choice			\\

\end{longtable}

\begin{scriptsize}
\begin{longtable}{ r c l r }


\\[0.01in]

\multicolumn{4}{l}{Module}			 \\
\\[0.01in]
$module$	& $ \rightarrow $ 	& \%module $mident$ $\{$ $tdefg$ ; $\}$ $\{$ $vdefg$ ; $\}$				&			\\
\\[0.01in]

\multicolumn{4}{l}{Type defn.}			 \\
\\[0.01in]
$tdefg$ 	& $ \rightarrow $	& \%data $qtycon$ $\{$ $tbind$ $\}$  = $\{$ $[$ $cdef$ $\{$ ; $cdef$ $\}$ $]$ $\}$	& algebraic type	\\
		& $ | $			& \%newtype $qtycon$ $qtycon$ $\{ tbind \}$ = $ty$					& newtype		\\
\\[0.01in]

\multicolumn{4}{l}{Constr. defn.}			 \\
\\[0.01in]
$cdef$		& $ \rightarrow $	& $qdcon$ $\{$ @ $tbind$ $\}$ $\{$ $aty$ $\}^{+}$ 					& 			\\
\\[0.01in]

\multicolumn{4}{l}{Value defn.}			 \\
\\[0.01in]
$vdefg$		& $ \rightarrow $	& \%rec $\{$ $vdef$ $\{$ ; $vdef$ $\}$ $\}$						& recursive		\\
		& $ | $			& $vdef$										& non-recursive		\\
$vdef$ 		& $ \rightarrow $	& $qvar$ :: $ty$ = $exp$								& 			\\
\\[0.01in]

\multicolumn{4}{l}{Atomic expr.}			 \\
\\[0.01in]
$aexp$		& $ \rightarrow $	& $qvar$										& variable		\\
		& $ | $			& $qdcon$										& data constructor	\\
		& $ | $			& $lit$											& literal		\\
		& $ | $			& ( $exp$ ) 										& nested expr.		\\
\\[0.01in]

\multicolumn{4}{l}{Expression}			 \\
\\[0.01in]
$exp$		& $ \rightarrow $	& $aexp$										& atomic expr.		\\
		& $ | $			& $aexp$ $\{$ $arg$ $\}^{+}$ 								& application		\\
		& $ | $			& $\backslash$ $\{$ $binder$ $\}$ -$>$ $exp$						& abstraction		\\
		& $ | $			& \%let	$vdefg$ \%in $exp$								& local definition	\\
		& $ | $			& \%case ( $aty$ ) $exp$ \%of $vbind$ $\{$ $alt$ $\{$ ; $alt$ $\}$ $\}$			& case expr.		\\
		& $ | $			& \%cast $exp$ $aty$									& type coercion		\\
		& $ | $			& \%note "  $\{$ $char$ $\}$ " $exp$							& expression note	\\
		& $ | $			& \%external ccal " $\{$ $char$ $\}$ " $aty$						& external reference	\\
		& $ | $			& \%dynexternal ccal $aty$								& external reference (dynamic)	\\
		& $ | $			& \%label " $\{$ $char$ $\}$ "								& external label	\\
\\[0.01in]

\multicolumn{4}{l}{Argument}			 \\
\\[0.01in]
$arg$		& $ \rightarrow $	& @ $aty$										& type argument		\\
		& $ | $			& $aexp$										& value argument	\\
\\[0.01in]

\multicolumn{4}{l}{Case alt}			 \\
\\[0.01in]
$alt$		& $ \rightarrow $	& $qdcon$ $\{$ @ $tbind$ $\}$ $\{$ $vbind$ $\}$ -$>$ $exp$				& constructor alternative \\
		& $ | $			& $lit$ -$>$ $exp$									& literal alternative 	\\
		& $ | $			& \%\_ -$>$ $exp$									& default alternative	\\
\\[0.01in]

\multicolumn{4}{l}{Binder}			 \\
\\[0.01in]
$binder$	& $ \rightarrow $	& @ $tbind$										& type binder		\\
		& $ | $			& $vbind$										& value binder		\\
\\[0.01in]

\multicolumn{4}{l}{Type binder}			 \\
\\[0.01in]
$tbind$		& $ \rightarrow $	& $tyvar$										& implicit of kind * 	\\
		& $ | $			& ( $tyvar$ :: $kind$ )									& explicitly kinded	\\
\\[0.01in]

\multicolumn{4}{l}{Value binder}			 \\
\\[0.01in]
$vbind$		& $ \rightarrow $	& ( $var$ :: $ty$ )									& \\
\\[0.01in]

\multicolumn{4}{l}{Literal}			 \\
\\[0.01in]
$lit$		& $ \rightarrow $	& ( $[$-$]$ $\{$ $digit$ $\}^{+}$ :: $ty$ )						& integer 		\\ 
		& $ | $			& ( $[$-$]$ $\{$ $digit$ $\}^{+}$ \% $\{$ $digit$ $\}^{+}$ :: $ty$ )			& rational		\\
		& $ | $			& ( ' $char$ ' :: $ty$ )								& character		\\
		& $ | $			& ( " $\{$ $char$ $\}$ " :: $ty$ )							& string		\\
\\[0.01in]

\multicolumn{4}{l}{Character}			 \\
\\[0.01in]
$char$		& $ \rightarrow $	& \multicolumn{2}{l}{Any ASCII character in range 0x20-0x7E except 0x22, 0x27, 0x5c}			 \\
		& $ | $			& $\backslash$x $hex$ $hex$								& ASCII code escape sequence \\
$hex$		& $ \rightarrow $	& 0 $|$ ... $|$ 9 $|$ a $|$ ... f							& \\
\\[0.01in]

\multicolumn{4}{l}{Atomic type}			 \\
\\[0.01in]
$aty$		& $ \rightarrow $	& $tyvar$										& type variable 	\\
		& $ | $			& $qtycon$										& type constructor	\\
		& $ | $ 		& ( $ty$ )										& nested type 		\\
\\[0.01in]

\multicolumn{4}{l}{Basic type}			 \\
\\[0.01in]
$bty$		& $ \rightarrow $	& $aty$											& atomic type		\\
		& $ | $			& $bty$ $aty$										& type application	\\
		& $ | $			& \%trans $aty$ $aty$									& transitive coercion 	\\
		& $ | $			& \%sym	$aty$										& symmetric coercion	\\
		& $ | $			& \%unsafe $aty$ $aty$									& unsafe coercion	\\
		& $ | $			& \%left $aty$										& left coercion		\\
		& $ | $			& \%right $aty$										& right coercion	\\
		& $ | $			& \%inst $aty$ $aty$									& instantiation coercion \\
\\[0.01in]

\multicolumn{4}{l}{Type}			 \\
\\[0.01in]
$ty$		& $ \rightarrow $	& $bty$											& basic type 		\\
		& $ | $			& \%forall $\{$ $tbind$ $\}^{+}$ . $ty$							& type abstraction	\\
		& $ | $			& $bty$ -$>$ $ty$									& arrow type construction \\
\\[0.01in]

\multicolumn{4}{l}{Atomic kind}			 \\
\\[0.01in]
$akind$		& $ \rightarrow $	& $*$											& lifted kind 		\\
		& $ | $			& \#											& unlifted kind 	\\
		& $ | $			& ?											& open kind 		\\
		& $ | $			& $bty$ :=: $bty$									& equality kind 	\\
		& $ | $			& ( $kind$ ) 										& nested kind 		\\
\\[0.01in]

\multicolumn{4}{l}{Kind}			 \\
\\[0.01in]
$kind$		& $ \rightarrow $	& $akind$										& atomic kind		\\
		& $ | $			& $akind$ -$>$ $kind$									& arrow kind		\\
\\[0.01in]

\multicolumn{4}{l}{Identifier}			 \\
\\[0.01in]
$mident$	& $ \rightarrow $	& $pname$ : $uname$									& module		\\
$tycon$		& $ \rightarrow $	& $uname$										& type constr.		\\
$qtycon$	& $ \rightarrow $	& $mident$ . $tycon$									& qualified type constr.\\
$tyvar$		& $ \rightarrow $	& $lname$										& type variable		\\
$dcon$		& $ \rightarrow $	& $uname$										& data constr.		\\
$qdcon$		& $ \rightarrow $	& $mident$ . $dcon$									& qualified data constr.\\
$var$		& $ \rightarrow $	& $lname$										& variable		\\
$qvar$		& $ \rightarrow $	& $[$ $mident$ . $]$ $var$								& optionally qualified variable\\
\\[0.01in]

\multicolumn{4}{l}{Name}			 \\
\\[0.01in]
$lname$		& $ \rightarrow $	& $lower$ $\{$ $namechar$ $\}$								& \\
$uname$		& $ \rightarrow $	& $upper$ $\{$ $namechar$ $\}$								& \\
$pname$		& $ \rightarrow $	& $\{$ $namechar$ $\}^{+}$								& \\
$namechar$	& $ \rightarrow $	& $lower$ $|$ $upper$ $|$ $digit$							& \\
$lower$		& $ \rightarrow $	& a$|$b$|$...$|$z$|$\_									& \\
$upper$		& $ \rightarrow $	& A$|$B$|$...$|$Z$|$									& \\
$digit$		& $ \rightarrow $	& 0$|$1$|$...$|$9									& \\
\\[0.01in]

\caption{Grammar for External-core}
\label{core}

\end{longtable}
\end{scriptsize}




% Z-Encoding
\clearpage
\section{Z-Encoding}
\label{zencoding}

The External-core identifiers are z-encoded. Meaning they use "z" and "Z" as prefixes
for certain symbols:

\begin{tabular}{ | l | l |}
\hline
Character 	& Code	\\
\hline
Tuples:		& 	\\

(\#\#)		& Z1H	\\
()		& Z0T	\\
(,,,)		& Z3T	\\

\hline
Constructors:	&	\\
(		& ZL	\\
)		& ZR	\\
{[}		& ZM	\\
{]}		& ZN	\\
:		& ZC	\\
Z		& ZZ	\\
\hline
Variables:	&	\\
z		& zz	\\	
\&		& za	\\
|		& zb	\\
\^{}		& zc	\\
\$		& zd	\\
=		& ze	\\
>		& zg	\\
\#		& zh	\\
.		& zi	\\
<		& zl	\\
-		& zm	\\
!		& zn	\\
+		& zp	\\
'		& zq	\\
\textbackslash	& zr	\\
/		& zs	\\
{*}		& zt	\\
\_		& zu	\\
\%		& zv	\\
c		& znnnU	\\
\hline
\end{tabular}



% JSON grammar
\clearpage
\section{Formal definition of JSON}
\label{jsongrammar}

\begin{scriptsize}
\begin{longtable}{ r c l r }
\multicolumn{4}{l}{Object}		\\
\\[0.01in]
$object$	& $ \rightarrow $ 	& \{ \}					& \\
 		& $ | $			& \{ $members$ \} 			& \\
$members$ 	& $ \rightarrow $	& $pair$				& \\
		& $ | $			& $pair$ , $members$ 			& \\
$pair$		& $ \rightarrow $	& $string$ : $value$ 			& \\
\\[0.01in]

\multicolumn{4}{l}{Array}		\\
$array$		& $ \rightarrow $	& [ ]					& \\
		& $ | $			& [ $elements$ ]			& \\
$elements$ 	& $ \rightarrow $	& $value$				& \\
		& $ | $			& $value$ , $elements$			& \\
\\[0.01in]

\multicolumn{4}{l}{Value}		\\
$value$		& $ \rightarrow $	& $string$				& \\
		& $ | $			& $number$				& \\
		& $ | $			& $object$				& \\
		& $ | $			& $array$				& \\
		& $ | $			& true					& \\
		& $ | $			& false					& \\
		& $ | $			& null					& \\
\\[0.01in]

\multicolumn{4}{l}{String}		\\
$string$	& $ \rightarrow $	& ""					& \\
		& $ | $			& " $chars$ "				& \\
$chars$		& $ \rightarrow $	& $char$				& \\
		& $ | $			& $char$ $chars$			& \\
$char$		& $ \rightarrow $	& any Unicode character except $"$ 	& \\ 
		&			& or $\backslash$ or control characters: & \\
		&			& $\backslash\backslash$		& \\
		&			& $\backslash /$ 			& \\
		&			& $\backslash b$ 			& \\
		& 			& $\backslash f$ 			& \\
		&			& $\backslash n$			& \\
		& 			& $\backslash r$ 			& \\
		&			& $\backslash t$ 			& \\
		& 			& $\backslash u$ four-hex digits\\
\\[0.01in]

\multicolumn{4}{l}{Number}		\\
$number$	& $ \rightarrow $ 	& $int$ 				& \\
		& $ | $			& $int$ $frac$				& \\
		& $ | $			& $int$ $exp$				& \\
		& $ | $			& $int$ $frac$ $exp$			& \\
$int$		& $ \rightarrow$ 	& $digit$				& \\
		& $ | $ 		& $digit1-9$ $digits$			& \\
		& $ | $ 		& - $digit$				& \\
		& $ | $ 		& - $digit1-9$ $digits$			& \\
$frac$ 		& $ \rightarrow $ 	& . $digits$ 				& \\
$exp$		& $ \rightarrow $ 	& $e$ $digits$ 				& \\
$digits$	& $ \rightarrow $ 	& $digit$				& \\
		& $ | $ 		& $digit$ $digits$			& \\
$e$		& $ \rightarrow $ 	& e					& \\
		& $ | $ 		& e+					& \\
		& $ | $ 		& e- 					& \\
		& $ | $ 		& E					& \\
		& $ | $ 		& E+					& \\
		& $ | $ 		& E-					& \\
\\[0.01in]

\caption{Grammar for JSON}
\label{json}
\end{longtable}

\end{scriptsize}




% JSCore grammar
\clearpage

\chapter{Formal definition of JSCore}
\label{jscoregrammar}


\begin{grammar}
<module> 	::= \{ ''\%module'' : <mident> , ''tdefg'' : [ <tdefg> ] , ''vdefg'' : [ <vdefg> ] \}
\end{grammar}

\paragraph{Type definitions}

\begin{grammar}
<tdefg> 	  ::= 	 \{ ''\%data'' : <qtycon> , ''tbind'' : [ <tbind> ], ''cdef'' : [ <cdef> ] \}						
		  \alt 	 \{ ''\%newtype'' : <qtycon> , ''qtycon'' : <qtycon> , ''tbind'' : [ <tbind> ] , ''ty'' : <ty> \} 	

\end{grammar}

\paragraph{Constructor definitions}

\begin{grammar}


<cdef>		  ::= 	 \{ ''qdcon'' : <qdcon> , ''tbind'' : [ <tbind>  ] , ''aty'' : [<aty>]$^{+}$ \} 				 			

\end{grammar}

\paragraph{Value definitions}

\begin{grammar}

<vdefg>		  ::= 	\{ ''\%rec'' : [ <vdef> ]$^{+}$ \}    							
		  \alt 	<vdef>
<vdef> 		  ::= 	\{ ''qvar'' : <qvar> , ''ty'' : <ty> , ''exp'' : <exp> \}

\end{grammar}

\paragraph{Atomic Expressions}
\begin{grammar}


<aexp>		  ::= 	 \{ ''qvar'' : <qvar> \}
		  \alt 	 \{ ''qdcon'' : <qdcon> \}
		  \alt 	 \{ ''lit'' : <lit> \}
		  \alt 	 \{ ''exp'' : <exp> \} 


\end{grammar}

\paragraph{Expressions}

\begin{grammar}

<exp>		  ::= 	 <aexp>
		  \alt 	 \{ ''aexp'' : <aexp> , ''args'' : [ <arg> ]$^{+}$ \} 				
		  \alt 	 \{ ''lambda'' : [ <binder> ] , ''exp'' : <exp> \}		
		  \alt 	 \{ ''\%let'' : <vdefg> , ''\%in'' : <exp> \}				
		  \alt 	 \{ ''\%case'' : <aty> , ''exp'' : <exp> , ''\%of'' : <vbind>, ''alt'' : [ <alt> ]$^{+}$ \}	
		  \alt 	 \{ ''\%cast'' : <exp> , ''aty'' : <aty>	\}		
		  \alt 	 \{ ''\%note'' : ''  \{ <char> \} '' , ''exp'' : <exp>	\}	
		  \alt 	 \{ ''\%external ccal'' : '' \{ <char> \} '' , ''aty'' : <aty> \}	
		  \alt 	 \{ ''\%dynexternal ccal'' : <aty> \}
		  \alt 	 \{ ''\%label'' : '' \{ <char> \} '' \}


\end{grammar}

\paragraph{Argument}

\begin{grammar}

<arg>		  ::= 	 \{ ''aty'' : <aty> \}											 
		  \alt 	 \{ ''aexp'' : <aexp> \}										


\end{grammar}

\paragraph{Case alternative}
\begin{grammar}

<alt>		  ::= 	 \{ ''qdcon'' : <qdcon> , ''tbind'' : [ <tbind> ] , ''vbind'' : [ <vbind> ] , ''exp'' : <exp> \}
		  \alt 			 \{ ''lit'' : <lit> , ''exp'' : <exp> \}
		  \alt 			 \{ ''\%\_'' : <exp> \}	


\end{grammar}

\paragraph{Binder}

\begin{grammar}

<binder>	  ::= 	\{ ''tbind'' : <tbind> \}		
		  \alt 	\{ ''vbind'' : <vbind> \}	


\end{grammar}

\paragraph{Type binder}

\begin{grammar}

<tbind>		  ::= 	 \{ ''tyvar'' : <tyvar> \}
		  \alt 	 \{ ''tyvar'' : <tyvar> , ''kind'' : <kind> \}	


\end{grammar}

\paragraph{Value binder}

\begin{grammar}

<vbind>		  ::= 	 \{ ''var'' : <var> , ''ty'' <ty> \} 										 

\end{grammar}

\paragraph{Literal}

\begin{grammar}


<lit>		  ::= 	<jsstring>		 
		  \alt 	<jsnumber>


<jsstring>	  ::= 	 ''''														 
		  \alt 	'' <jschars> ''		
											 
<jschars>	  ::= 	<jschar>
		  \alt 	<jschar> <jschars>
												 
<jschar>	  ::= 	 See definition below

<jsnumber>	  ::=  	<jsint>
		  \alt 	<jsint> <jsfrac>
		  \alt 	<jsint> <jsexp>
		  \alt 	<jsint> <jsfrac> <jsexp>
											 
<jsint>		  ::= 	<jsdigit>
		  \alt  <jsdigit1-9> <jsdigits> 
		  \alt  - <jsdigit>
		  \alt  - <jsdigit1-9> <jsdigits>

<jsfrac> 	  ::=  	. <jsdigits>

<jsexp>		  ::=  	<jse> <jsdigits>

<jsdigits>	  ::=  	<jsdigit> 
		  \alt  <jsdigit> <jsdigits>

<jse>		  ::=  	e
		  \alt  e+		 
		  \alt  e- 		 
		  \alt  E		 
		  \alt  E+		 
		  \alt  E-

\end{grammar}

\paragraph{Atomic Type}

\begin{grammar}
<aty>		  ::= 	 \{ ''tyvar'' : <tyvar> \}
		  \alt 	 \{ ''qtycon'' : <qtycon> \}
		  \alt   \{ ''ty'' : <ty> \}

\end{grammar}

\paragraph{Basic Type}

\begin{grammar}
<bty>		  ::= 	 <aty>
		  \alt 	 \{ ''bty'' : <bty> , ''aty'' , <aty> \}
		  \alt 	 \{ ''\%trans'' : <aty> , ''aty'' : <aty> \}
		  \alt 	 \{ ''\%sym'' : <aty> \}
		  \alt 	 \{ ''\%unsafe'' : <aty> , ''aty'' : <aty> \}	
		  \alt 	 \{ ''\%left'' : <aty> \}
		  \alt 	 \{ ''\%right'' : <aty> \}	
		  \alt 	 \{ ''\%inst'' : <aty> , ''aty'' : <aty> \}


\end{grammar}

\paragraph{Type}

\begin{grammar}

<ty>		  ::= 	 <bty>
		  \alt 	 \{ ''\%forall'' :  [ <tbind> ]$^+$ , ''ty'' : <ty> \}	
		  \alt 	 \{ ''bty'' <bty> , ''ty'' : <ty> \} 


\end{grammar}

\paragraph{Atomic Kind}

\begin{grammar}

<akind>		  ::= 	 *	
		  \alt 	 \#
		  \alt 	 ?	
		  \alt 	 \{ ''bty'' : <bty< , ''bty'' : <bty> \}
		  \alt 	 \{ ''kind'' : <kind> \}	


\end{grammar}

\paragraph{Kind}

\begin{grammar}

<kind>		  ::= 	\{ ''akind'' : <akind> \}					
		  \alt 	\{ ''akind'' : <akind> , ''kind'' : <kind> \}		

<mident>	  ::= 	 '' <pname> : <uname> ''
	
<tycon>		  ::= 	 '' <uname> ''
		
<qtycon>	  ::= 	 '' <mident> . <tycon> ''

<tyvar>		  ::= 	 '' <lname> ''
	
<dcon>		  ::= 	 '' <uname> ''
	
<qdcon>		  ::= 	 '' <mident> . <dcon> ''

<var>		  ::= 	 '' <lname> ''

<qvar>		  ::= 	 '' || <mident> . || <var> ''

<lname>		  ::= 	 <lower> \{ <namechar> \}
 
<uname>		  ::= 	 <upper> \{ <namechar> \}

<pname>		  ::= 	 \{ <namechar> \}$^+$

<namechar>	  ::= 	 <lower> | <upper> | <digit>

<lower>		  ::= 	 a|b|...|z|\_

<upper>		  ::= 	 A|B|...|Z|

<digit>		  ::= 	 0|1|...|9									 


\end{grammar}




% Test programs
\clearpage
\section{Test programs}
\label{testprograms}

\subsection{helloworld}

\lstinputlisting[language=Haskell]{"../interpreter/tests/helloworld/helloworld.hs"}

\subsection{helloworld2}

\lstinputlisting[language=Haskell]{"../interpreter/tests/helloworld2/helloworld2.hs"}

\subsection{factorial}

\lstinputlisting[language=Haskell]{"../interpreter/tests/factorial/factorial.hs"}

\subsection{fibonacci}

\lstinputlisting[language=Haskell]{"../interpreter/tests/fibonacci/fibonacci.hs"}




% JSCore program graphs
%\clearpage
%\section{Graphs of programs in JSCore}
\label{graphs}

\begin{sidewaysfigure}
\begin{figure}[H]
\includegraphics[width=\textwidth]{../interpreter/tests/helloworld/helloworld.png}
\caption{Example program translated to JSCore}
\label{helloworldgraph}
\end{figure}
\end{sidewaysfigure}

\begin{comment}

\begin{sidewaysfigure}
\begin{figure}[H]
\includegraphics[width=\textwidth]{../interpreter/tests/factorial/factorial.png}
\caption{Example program translated to JSCore}
\label{helloworldgraph}
\end{figure}
\end{sidewaysfigure}

\end{comment}


% References
\clearpage
\bibliographystyle{plain}
\bibliography{papers}

\end{document}
